\documentclass[12pt]{article}

\setlength\parindent{0pt}
\newcommand{\myt}[1]{\textbf{\underline{#1}}}

\usepackage{mathtools}
\usepackage{amssymb}

\title{\vspace{-15ex}CS241 Lecture 10\vspace{-1ex}}
\date{June 8th 2015}
\author{Graham Cooper}

\begin{document}
	\maketitle
	
	\myt{Recall:}\\
	A DFA is a 5-tuble $(\Sigma, Q, q_0, A. \delta)$\\
	\begin{itemize}
		\item $\Sigma, Q$ finite, non-empty sets (alphabet, states)
		\item $q_0 \in Q$ (start)
		\item $A \subseteq Q$ (accepting)
		\item $\delta:  Q \times \Sigma \implies Q$ (transition)
	\end{itemize}
	It accepts a word w if $\delta^*(q_0, w) \in A$\\
	
	\section*{Implementing a DFA}
	
	\begin{verbatim}
	int state = q_0
	char c
	while not EOF do
	-- read c
	-- case state of
	-- -- q_0: case c of
	-- -- -- a: state = ...
	-- -- -- b: state= ...
	-- -- -- ...
	-- -- q_1: case c of
	-- -- -- a: state = ...
	-- -- -- b: state = ...
	-- -- -- ...
	...
	-- -- q_n: case c of
	-- -- -- a: state = ...
	-- -- -- b: state = ...
	-- -- -- ...
	-- end case
	end while
	return true if state is in A
	\end{verbatim}
	
	Or use a lookup table
	
	\begin{tabular}{c | c c | c | c }
		& states & & & \\ \hline
		characters & & & next state & \\ \hline
	\end{tabular}
	
	\section*{DFA's with actions}
	We can attach computations to the arcs of a DFA.\\
	
	eg: L = \{binary numbers with no loading 0's\}\\
	- Compute the value of the number\\
	- 1(0$|$1)*$|$0\\
	SET PICTURE\\
	
	What do we gain by making DFA's more complex?\\
	eg: \\
	
	L = $\{w \in \{a,b\}* | w ends with abb \}$\\
	SEE PICTURE\\
	
	What if we allowed more than one arc with the same char from the same state?\\
	
	What would this mean?\\
	Machine chooses one direction or the other (i.e. the machine is non-deterministic). Accept if some set of choices leads to an accepting state.
	
	With non-determinism, the machine becomes: SEE PICTURE\\
	
	The machine gueses to stay in the first state until it reaches the final abb, then transitions to accepting.\\
	- NFA's often simpler than DFA's\\
	
	Formally: - An NFA is a 5-tuble $(\Sigma, Q, q_0, A, \delta)$ where:
	\begin{itemize}
		\item $\Sigma$ is a finite, non-empty set (Alphabet)
		\item Q is a finite, non-empty set (states)
		\item $q_0 \in Q$ (start state)
		\item $A \subseteq Q$ (accepting states)
		\item $\delta: Q \times \Sigma \implies$ subsets of Q ($2^Q$) - non-determinism
	\end{itemize}
	
	Want to acept if some path through the NFA leads to acceptance for the given word (reject if none do)\\
	
	$\delta^*$ for NFA's: set of states X $\Sigma \implies$ set of states\\
	
	$\delta^*(qs, \epsilon)$ = qs\\
	qs (spoken as multiple q states)\\
	$\delta^*(qs, cw)$ = $\bigcup_{q \in qs}(\delta(q,c), w)$\\
	
	Then accept w if $\delta^*(\{q_0\},w) \bigcap A \neq \emptyset$\\
	
	NFA simulation procedure:\\
	States $\leftarrow \{q_0\}$\\
	while not eof do\\
	-- read ch\\
	-- states $\leftarrow \bigcup_{q \in states}(\delta(q,ch))$\\
	end do
	return states $\cap A \neq \emptyset$\\
	
	Using the NFA:\\
	$\rightarrow 1$(loops on a,b)$ \overset{a}{\rightarrow} 2 \overset{b}{\rightarrow} 3 \overset{b}{\rightarrow} 4$ accept\\
	simulate baabb\\
	\begin{tabular}{c | c | c c}
		Already read input & Unread Input & states & rename \\ \hline
		$\epsilon$ & baabb & \{1\} & A\\
		b & aabb & \{1\} & A\\
		ba & abb & \{1,2\} & B\\
		baa & bb & \{1,2\} & B\\
		baab & b & \{1,3\} & C\\
		baabb & $\epsilon$ & \{1,4\} & D\\
	\end{tabular}
	
	\{1,4\} $\cap$ \{4\} = \{4\} $\neq \emptyset$, accept!\\
	See picture to go along with this\\
	We can find the DFA from following the renamed states.(subset construction)\\
	If you give each set of states a name and use those as the states, every NFA becomes a DFA!\\
	
	More complex example:\\
	
	L = \{cab\} $\cup$ \{strings over \{a,b,c\} with an even number of a's\}\\
	
	
	
	
	
	
	
	
\end{document}
