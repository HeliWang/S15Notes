\documentclass[12pt]{article}

\setlength\parindent{0pt}
\newcommand{\myt}[1]{\textbf{\underline{#1}}}

\usepackage{mathtools}
\usepackage{amssymb}
\usepackage{listings}
\usepackage{mips}

\title{\vspace{-15ex}CS241 - Lecture 4\vspace{-1ex}}
\date{May 13th, 2015}
\author{Graham Cooper}

\begin{document}
	\maketitle
	Recall Labels
	\lstset{language=[mips]Assembler}
	\begin{lstlisting}
	0 lis $2
	4 .word 13
	8 add $3, $0, $0
	c top: add $3, $3, $2 ; top = 0x0c
	10 lis $1
	14 .word 1
	18 sub $2, $2, $1
	1c sub $2, $2, $1
	1c bne $2, $0, top
	20 jr $31
	\end{lstlisting}
	
	\section*{Procedures in MIPS}
	\myt{Two problems to solve}
	\begin{enumerate} 
		\item Call and return - how to transfer control into and out of a procedure. What if the proc calls another proc? - Parameters?
		\item Registers - What if a procedure overwrites regs we were using? - Data is lost.
	\end{enumerate}
	
	\subsection*{2)}
	We could reverse some registers for the procedure f, and some for the mainline - then they won't interfere.\\
	- What if f calls g, g calls h,.... eventually run out of registers.\\
	\underline{Instead:}\\
	Guarantee that procedures leave registers unchanged when complete.\\
	How? - Use RAM. - Which RAM?\\
	- How do we keep procedures from using the same RAM?\\\\
	Diagram of RAM:\\
	
	\begin{tabular}{c| c |}
		0 & Your Program \\ \hline
		  & FREE RAM \\
	\end{tabular} 
	
	Allocate RAM from either the top or the bottom of free RAM.\\
	Keep Track of which RAM is in use and which isn't\\
	MIPS Machine helps us out:\\
	\$30 initicalized by the loader to just past the last word of free RAM\\
	
	Can use \$30 as a bookmark to seperate used and unused RAM, if we allocate from the bottom of free RAM\\
	
	Eg. Procedures f,g,h.\\
	Say f calls g,\\
	g calls h\\
	h returns\\
	g returns\\
	f returns\\
	
	Each procedure stores in RAM the registers it wants to use and restores the original values on return.\\
	
	\begin{tabular}{| c |}
		\hline
		CODE \\ \hline
		\\ 
		\\
		\\ \hline
		h's regs \\ \hline
		g's regs \\ \hline
		f's regs \\ \hline
	\end{tabular}
	
	As functions are called, register \$30 goes towards code, and as things returns the value goes away from the code\\
	
	RAM is used in LIFO order - as a stack\\
	\myt{\$30}
	\begin{itemize}
		\item the \underline{stack pointer}
		\item contains the address of the top of the stack.
	\end{itemize}
	
	\subsection*{Templates for Procedures}
	(assume f uses \$2 and \$3)
	\lstset{language=[mips]Assembler}
	\begin{lstlisting}
	0 f: sw $2, -4($30)
	4 sw $3, -8($30)
	8 lis $3
	c .word 8
	10 sub $30, $30, $3
	...
	...
	n add $30, $30, $3
	n+4 lw $3, -8($30)
	n+8 lw $2, -4($30)
	n+c ???
	\end{lstlisting}
	Usage:\\
	0-4: Push registers f will modify onto the stack\\
	8-10: Decrement \$30
	10-n: body of the function\\
	n: assuming \$3 is still 8, increment \$30\\
	n+4,n+8: popping the registers from the stack\\
	n+c: return?
	
	\subsection*{1)}
	Call and Return\\
	\myt{call}
	\lstset{language=[mips]Assembler}
	\begin{lstlisting}
	main: ...
	main: ...
	main: ...
	n lis $5
	n+4 .word f
	n+8 jr $5
	\end{lstlisting}
	n+4: address of line labelled 'f'\\
	n+8: jump to that line\\
	
	\myt{return}\\
	- Need to set PC to the line after the jr\\
	- How does f know where that is?\\
	
	\myt{Solution}\\
	- jalr instruction ("Jump and link to reg")\\
	- like jr but sets \$31 to the address of the next instruction, before jumping.\\
	So replace jr \$5 above with jalr \$5 then return is jr \$31\\
	
	\underline{Q:} jalr overwrites \$31 - then how can we return to the loader? And what if f calls g?\\
	
	\underline{A:} need to save \$31 on the stack first, and restore it when the call returns.\\
	\lstset{language=[mips]Assembler}
	\begin{lstlisting}
	main: ...
	main: ...
	main: ...
	n lis $5
	n+4 .word f
	n+8 sw $31, -4($30)
	n+c lis $31
	n+10 .word 4
	n+14 sub $30, $30, $31
	n+18 jalr $5
	n+1c lis $31
	n+20 .word 4
	n+24 add $30, $30, $31
	n+28 lw $31, -4($30)
	...
	...
	n+i jr $31
	\end{lstlisting}
	n+8: push \$31 onto the stack\\
	n+8,n+14: Decrement stack pointer\\
	n+1c,n+20: Increment stack pointer\\
	
	f:
	\begin{itemize}
		\item push regs
		\item decrement \$30
		\item body
		\item increment \$30
		\item Pop Regs
		\item Return
	\end{itemize}
	
	\subsection*{Parameters and Results}
	- Generally use registers (DOCUMENT!!)\\
	- if too many parameters use he stack\\
	
	Eg: Sum 1 to N\\
	
	; sum 1 to N - adds, 1,...N\\
	; Registers: \\
	; \$1 - working - save this reg\\
	; \$2 - input - save this reg\\
	; \$3 - output - do not save\\
	
	Sum 1 to N:
	\begin{lstlisting}
	sw $1, -4($30) ; Push $1 and $2
	sw $2, -8($30) 
	lis $1
	.word 8
	sub $30, $30, $1
	add $3, $0, $0
	top: add $3, $3, $2
	lis $1
	.word 1
	sub $2, $2, $1
	bne $2, $0, top
	lis $1
	.word 8
	add $30, $30, $1
	lw $2, -8($30)
	lw $1, -4($30)
	jr $31
	\end{lstlisting}
	
	\subsection*{Recursion}
	- No extra machinery is needed!!!! YAY\\
	- If registers, parameters and stack are managed properly, recursion will just work\\
	
	\subsection*{Output:}
	Use sw to store a word at location 0xffff000c\\
	- Least significant byte will be printed\\
	-To print CS\textbackslash n
	\begin{lstlisting}
	lis $1
	.word 0xffff000c
	lis $2
	.word 67 ; c
	sw $2, 0($1)
	lis $2
	.word 83 ; s
	sw $2, 0($1)
	lis $2
	.word 10 ; backslash n
	sw $2, 0($1)
	jr $31
	\end{lstlisting}
	
	
\end{document}
