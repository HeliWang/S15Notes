\documentclass[12pt]{article}

\setlength\parindent{0pt}
\newcommand{\myt}[1]{\textbf{\underline{#1}}}

\usepackage{mathtools}
\usepackage{amssymb}
\usepackage{geometry}
\geometry{
	a4paper,
	total={210mm,297mm},
	left=5mm,
	right=20mm,
	top=20mm,
	bottom=20mm,
}
\title{\vspace{-15ex}Lec 16 CS241\vspace{-1ex}}
\date{June 29th, 2015}
\author{Graham Cooper}

\begin{document}
	\maketitle
	
	\myt{Choice:}\\
	\begin{enumerate}
		\item Shift symbol from input to stack
		\item reduce top of stack by grammar rule
	\end{enumerate}
	
	\section*{Theorem (Donald Knuth, 1965)}
	- create tex, became latex
	- currently writing the art of computer programming
	
	\myt{Theorem:}\\
	The set \{wa $|$ $\exists$x, S' $\implies$*wax \}\\
	w is the stabck\\
	a is the next characer\\
	is a regular language.\\
	
	If it is a regular language, then it can be described by a DFA.\\
	- use the DFA to make shift/reduce decisions\\
	
	\subsection*{LR Parsing}
	\begin{itemize}
		\item left-to-right scan
		\item rightmost derivations
	\end{itemize}
	
	\myt{Example:}
	S' $\rightarrow$ $\vdash$ E $\dashv$\\
	E $\rightarrow$ E + T\\
	E $\rightarrow$ T\\
	T $\rightarrow$ id\\
	See pretty picture 1 in notes\\
	
	\subsection*{LR(0) machine (simplest)}
	\myt{Definition:} An item is a production with a dot ($\cdot$) somewhere on the right hand side.\\
	(indiates partially completed rule)\\
	
	\begin{itemize}
		\item begin with a state with the starting state production rule and the dot at the beginning of input
		\item Label an arc with the symbol that follows the dot ; advance the dot in the next state
		\item If the dot precedes a non-terminal A, add al productions with A on the LHS to the state, dot in the leftmost position.
		\item Repeat until we get all of the transitions we can
	\end{itemize}
	
	\subsubsection*{Using the machine}
	\begin{itemize}
		\item Start in the start state with empty stack
		\item Shifting \\
		- shift char from input to stack \\
		- follow transiition for that char to next state \\
		- if no transition, error, or reduce
		\item Reducing\\
		- "Reduce states" have only one item and the dot is rightmost\\
		- called a complete item\\
		- reduce by the rule in the state\\
		- reduce: pop RHS off the stack, backgrack size(RHS) states in the DFA, push LHS, follow shift transition for the LHS
		\item Backtracking the DFA\\
		- must remember the DFA states\\
		- Push the DFA states onto the stack as well.
	\end{itemize}
	
	\begin{tabular}{c | c | c | c}
		Stack & Read & unread & Action \\ \hline
		1 & $\epsilon$ & $\vdash$id+id+id$\dashv$ & S2 (shift and go to 2) \\
		1 $\vdash$ 2 & $\vdash$ & id+id+id$\dashv$ & S6 (Shift and go to 6) \\
		1 $\vdash$ 2 id 6 & $\vdash$id & +id+id$\dashv$ & RT $\rightarrow$ id (Pop 1 symbol and 1 state) Now in state 2, Push T go to 5\\
		1 $\vdash$ 2 T 5 & $\vdash$id & +id+id$\dashv$ & R, E $\rightarrow$ T pop 1 sym, pop 1 state, push E goto 3\\
		1 $\vdash$ 2 E 3 & $\vdash$id & +id+id$\dashv$ & S7 \\ 
		1 $\vdash$ 2 E 3 + 7 & $\vdash$ id + & id+id$\dashv$ & S6 \\
		1 $\vdash$ 2 E 3 + 7 + 6 & $\vdash$ id + id & +id$\dashv$ & R, T $\rightarrow$ id, goto 8\\
		1 $\vdash$ 2 E 3 + 7 T 8 & $\vdash$ id+id & +id$\dashv$ & R, E $\rightarrow$ E + T (Pop 3 sym, 3 states, goto 2)\\
		1 $\vdash$ 2 E 3 & $\vdash$id + id & +id $\dashv$ & S7 \\
		1 $\vdash$ 2 E 3 + 7 & $\vdash$id + id + & id $\dashv$ & S6 \\
		1 $\vdash$ 2 E 3 + 7 id 6 & $\vdash$id+id+id & $\dashv$ & R, T $\rightarrow$ id, goto 8 \\
		1 $\vdash$ 2 E 3 + 7 T 8 & $\vdash$id+id+id & $\dashv$ & R, E $\rightarrow$ E + T goto3 \\
		1 $\vdash$ 2 E 3 & $\vdash$id+id+id & $\dashv$ & S4\\
		1 $\vdash$ 2 E 3 $\dashv$ 4 & $\vdash$id+id+id$\dashv$ & $\epsilon$ & Accept\\
	\end{tabular}
	
	What can go wrong?\\
	What if the state looks like this:\\
	A $\rightarrow$ $\alpha \cdot$ c $\beta$\\
	B $\rightarrow$ $\gamma$ $\cdot$\\
	Shift c or reduce B $\rightarrow$ $\gamma$\\
	This is a shift-reduce conflict!\\
	
	A $\rightarrow$ $\alpha$ $\cdot$\\
	B $\rightarrow$ $\beta$ $\cdot$ \\
	reduce-reduce conflict\\
	
	Whenever a complete item A $\rightarrow$ $\alpha \cdot$ is not alone in a state there is a conflict and the grammar is not LR(0)\\
	
	\myt{Example:} Right-associative\\
	S' $\rightarrow$ $\vdash$ E $\dashv$\\
	E $\rightarrow$ T + E\\
	E $\rightarrow$ T \\
	T $\rightarrow$ id\\
	
	See pretty picture 2 in notes\\
	
	\myt{Example:}\\
	$\epsilon$1 shift $->$ $\vdash$2 shift $->$ $\vdash$id 6 reduce $->$ $\vdash$ T 5\\
	
	Should we reduce E$\rightarrow$ T?\\
	Depends: If input is $\vdash$ id $\dashv$ then YES, otherwise, no!\\
	
	Add a lookahead to fix the conflict\\
	
	For each A $\rightarrow$ $\alpha$ $\cdot$, attach Follow(A)\\
	Follow(E) = \{$\dashv$\}\\
	Follow(A) = \{+, $\dashv$\}\\
	
	Interpretation:\\
	A reduce action \\
	A $\rightarrow$ $\alpha \cdot$ X (X = follow(A))\\
	only applies if the next char is in X.\\
	
	So E $\rightarrow$ T $\cdot$ applies when the next char is $\dashv$\\
	E $\rightarrow$ T $\cdot$ + E applies when next char is +\\
	
	Conflict resolved!\\
	Result is called an SLR(1) parser. = Simple LR with 1 char lookahead\\
	
	SLR(1) resolves many, but not all conflicts\\
	
	LR(1) - more powerful than SLR(1)\\
	- PRoduces many more states\\
	
	\section*{Building a Parse Tree}
	\subsection*{Top-Down}
	\begin{tabular}{c | c }
		$\dashv$ S & S $\rightarrow$ A y B \\
		$\dashv$ B y A & keep S make the new nodes its children\\
	\end{tabular}
	
	\subsection*{Bottom-up}
	\begin{tabular}{c | c }
		$\dashv$ab & Reduce A $\rightarrow$ a b\\
		$\dashv$A & Use A as parent, make a b as children\\
	\end{tabular}
	
	
	
	
	
\end{document}
