\documentclass[12pt]{article}

\setlength\parindent{0pt}
\newcommand{\myt}[1]{\textbf{\underline{#1}}}

\usepackage{mathtools}
\usepackage{amssymb}

\title{\vspace{-15ex}Lec17 CS241\vspace{-1ex}}
\date{July 6th, 2015}
\author{Graham Cooper}

\begin{document}
	\maketitle
	
	\section*{Context-Sensitive Analysis (also called Semantic Analysis)}
	What properties of valid programs cannot be enforced by CFG's?\\
	\begin{itemize}
		\item Declaration before use
		\item Scope Correctness
		\item Type Checking
		\item etc...
	\end{itemize}
	
	To solve these, we need more complex programs\\
	The next language class: Context-sensitive languages\\
	(specified by context-sensitive grammars)\\
	
	-Ad hoc approach actually more useful\\
	- Traverse the parse tree \\
	
	Eg. Parse tree\\
	class Tree \{\\
	public:\\
	- string rule;\\
	- vector$<$Tree *$>$ children;\\
	\};\\
	The string rule is the grammar rule "term id" or the token type nad lexeme: "id xyz"\\
	
	\subsection*{Traversal Example: Print the tree}
	\begin{verbatim}
	void print(Tree &t){
	-- //print t.rule
	-- foo(vector<Tree *> i ; iterator i = t.children.begin(); i != t.children.end; ++i){
	-- -- print (**i);
	-- }
	}
	\end{verbatim}
	
	\subsection*{Error Checking}
	What semantic analysis is needed for WLP4?\\
	What errors need to be caught?\\
	\begin{itemize}
		\item variables and procedures, used but not declared
		\item variable/procedure duplicates
		\item scope?
		\item type errors
		\item For now, assume that wain is the only procedure
	\end{itemize}
	
	\subsubsection*{Decleration errors}
	multiple/missing declarations, how do we do this, we've seen this before\\
	Construct a symbol table!!!!\\
	\begin{itemize}
		\item traverse the parse tree to collect all declarations.
		\item For each node corresponding to rule dcl $\rightarrow$ TYPE ID
		\item Extract the name and type and add to the symbol table
		\item if name is alrady in the symboltable, ERROR
		\item Multiple Declarations now checked
	\end{itemize}
	
	Traverse again!\\
	\begin{itemize}
		\item Check for rules of the form factor $\rightarrow$ ID and lvalue $\rightarrow$ ID\\
		\item if the ID's name is not in the symbol table, then ERROR
		\item undeclared variables are now checked
	\end{itemize}
	
	The above two passes could be merged.\\
	
	\subsection*{Symbol Table Implementation - global variable}
	map $<$string, string$>$ symbolTable\\
	(name, type)\
	But:\\
	\begin{itemize}
		\item doesn't take scope into account
		\item doesnt check procedure declerations
	\end{itemize}
	
	\myt{Issue:}\\
	\begin{verbatim}
	int f() {
	-- int x = 0;
	-- return 1;
	}
	
	int wain (int a, int b){
	-- int x = 0;
	-- return 1;
	}
	\end{verbatim}
	This will give an error (only using the above two passes) and it should be OK!\\
	
	Must forbid duplicate definitions in the same procedure, but allow them in different procedures\\
	
	\myt{Also:} \\
	\begin{verbatim}
	int f() { ... }
	int f() { ... }
	\end{verbatim}
	
	Must also forbid duplicate procedure definitions\\
	
	\myt{Sol'n}:\\
	\begin{itemize}
		\item Have a seperate symbol table for each procedure
		\item Have one "Top-Level" symbol table that collects all procedure names
	\end{itemize}
	
	map $<$string, map$<$string, string$>$ $>$ topSymtable;\\
	(proc name, procs symboltable)\\
	When traversing:\\
	When the rule is \\
	procedure $\rightarrow$ ...\\
	main $\rightarrow$ ...\\
	- these mean there is a new procedure\\
	- make sure it is not already in the symbol table, and if not, create a new entry\\
	- if not, create a new entry\\
	- May want a global variable called "currentPRoc"\\
	- Tells you which procedure you are currently in\\
	- keep it up to date\\
	
	For vars\\
	- We store the declared type in the symbol table\\
	Is there type into for procedures?\\
	Yes - Signature
	- Note: All procedures in WLP4 return int, so the signature is the sequence of param types\\
	- Store signature in top-level table\\
	
	map$<$ string, pair$<$vector $<$string$>$, map$<$string,string$>$ $>$ $>$ topSymtbl;\\
	(proc-name signature, local Symbol table)\\
	
	To compute the signature:\\
	Traverse nodes of the form 
	- paramlist $\rightarrow$ dcl\\
	- paramlist $\rightarrow$ dcl COMMA paramlist\\
	
	\section*{Types}
	Why do programming languages have types?\\
	\begin{itemize}
		\item Prevent Errors
		\item allow us to assign an interpretation to the contents of some address and to remember the interpretation.\\
	\end{itemize}
	
	eg. int *a = NULL; a denotes a pair\\
	a = 7; X attempt to stare an int where a pointer is denoted\\
	
	To check type ocrrectness\\
	\begin{itemize}
		\item find a type for each var/expr
		\item ensure the "rules are paplied properly
	\end{itemize}
	
	Another tree traversal:\\
	\begin{verbatim}
	string typeOf(Tree &t){
	-- for each c in t.children, 
	-- --compute typeOF(c)
	-- determine typeOF(t) from types of children, based on t.rule
	}
	\end{verbatim}
	
	\subsection {Example: ID}
	- get its type from symbol table\\
	
	\underline{$<$id.name, $\tau$ $>$ $\in$ declarations}\\
	id : $\tau$\\
	
	\begin{verbatim}
	string typeof(Tree &t){
	-- if t.rule == "ID name"
	-- -- return symtbl.lookup(name)
	}
	\end{verbatim}
	
	\subsection*{Singleton Productions}
	expr $\rightarrow$ term\\
	term $\rightarrow$ factor\\
	factor $\rightarrow$ id\\
	
	\underline{Type of LHS = type of RHS}\\
	NUM: int ... NULL : int *\\
	
	\underline{E : $\tau$}\\
	(E) : $\tau$\\
	
	E has type $\tau$ then (E) has type $\tau$\\
	
	
	
	
\end{document}










