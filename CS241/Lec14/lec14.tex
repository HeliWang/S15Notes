\documentclass[12pt]{article}

\setlength\parindent{0pt}
\newcommand{\myt}[1]{\textbf{\underline{#1}}}

\usepackage{mathtools}
\usepackage{amssymb}

\title{\vspace{-15ex}CS 241 Lecture 14\vspace{-1ex}}
\date{June 22nd, 2015}
\author{Graham Cooper}

\begin{document}
	\maketitle
	
	\section*{Top-Down Parsing}
	
	$$S \implies \alpha_1 \implies ... \implies \alpha_n \implies w$$
	
	Invariant: Consumed input + reverse(stack contents) = $\alpha_i$\\
	
	\myt{Example:}\\
	S $\rightarrow$ AyB\\
	A $\rightarrow$ ab\\
	A $\rightarrow$ cd\\
	B $\rightarrow$ z\\
	B $\rightarrow$ wx\\
	
	For simplicity, we use augumented grammars for parsing, ie. invent two new symbols $\vdash, \dashv$, new start symbol S'\\
	
	\begin{enumerate}
		\item S' $\rightarrow \vdash S \dashv$
		\item S $\rightarrow$ AyB
		\item A $\rightarrow$ ab
		\item A $\rightarrow$ cd
		\item B $\rightarrow z$
		\item B $\rightarrow$ wx
	\end{enumerate}
	
	\begin{tabular}{c | c | c | c }
		Stack & Read Input & Unread Input & Action \\ \hline
		S' & $\epsilon$ & $\vdash abywx \dashv$ & Pop S'; push $\dashv, S, \vdash$ \\
		$\dashv S \vdash$ & $\epsilon$ & $\vdash abywx \dashv$ & Match $\vdash$\\
		$\dashv S$ & $\vdash$ & abywx$\dashv$ & Pop S; Push B,y,A\\
		$\dashv ByA$ & $\vdash$ & abywx$\dashv$ & PopA; push b,a\\
		$\dashv Byba$ & $\vdash$ & abywx$\dashv$ & Match a \\
		$\dashv$ Byb & $\vdash a$ & bywx$\dashv$ &  Match b\\
		$\dashv$ By & $\vdash ab$ & ywx$\dashv$ & Match y\\
		$\dashv$ B & $\vdash aby$ & wx$\dashv$ & Pop B; push x,w\\
		$\dashv$ xw & $\vdash aby$ & wx$\dashv$ & Match w \\
		$\dashv$ x & $\vdash$abyw & x$\dashv$ & Match x \\
		$\dashv$ & $\vdash$abywx & $\dashv$ & Match $\dashv$\\
		& $\vdash$abywx$\dashv$ & $\epsilon$ & Accept\\ 
	\end{tabular}

	When top of stack (TOS) is a terminal pop and match against input.\\
	When TOS is a non-terminal A, popA and push $\alpha^R$ ($\alpha$ reversed), where A $\rightarrow \alpha$ is a grammar rule. \\
	Accept when stack and input are both empty\\
	
	\myt{BUT} what if there is $>$1 production with A on the LHS? How can we know which one to pick?\\
	
	\begin{itemize}
		\item Brute force (try all compinations until one works)
		\item Our solution: Use the next symbol of input (look ahead) to help decide (predictor table)
	\end{itemize}
	
	Construct a predictor table!\\
	- Given a non-terminal on the stack and input symbol, tell us which production to use.\\
	
	\begin{enumerate}
	\item S' $\rightarrow \vdash S \dashv$
	\item S $\rightarrow$ AyB
	\item A $\rightarrow$ ab
	\item A $\rightarrow$ cd
	\item B $\rightarrow z$
	\item B $\rightarrow$ wx
	\end{enumerate}
	
	\begin{tabular}{c | c  c  c  c  c  c  c  c  c c}
		& $\vdash$ & a & b & c & d & w & x & y & z & $\dashv$ \\ \hline
		S' & 1 & & & & & & & & & \\
		S & & 2 & & 2 & & & & & & \\
		A & & 3 & & 4 & & & & & & \\
		B & & & & & & 6 & & 5 & & \\ 
	\end{tabular}
	
	Empty cell = parse error\\
	
	Descriptive: "Parse error at row,col: expecting one of "chars \\
	\underline{for which curren top of stack has entries}\\
	
	What if $A \rightarrow$ ad, then you would have multiple states in a single table column\\
	
	\myt{What if a cell contains more than one entry?}\\
	THIS DOESN'T WORK...\\
	
	A grammar is called LL(1) if each cell of the predictor table has at most one entry. IF we have this situation then this table will work.\\
	
	LL(1) = left-to-right scan of input leftmost derivations produced 1 symbol of look ahead.\\
	That was way to long to be just LL...\\
	
	\section*{Automaticall computing the predictor table}
	
	Predict(A, a) gives you the rules that apply when A is on the stack and a is the next input character\\
	
	Predict(A, a) = \{A $\rightarrow \beta$ $|$ $a \in$ First($\beta$) \}\\
	First($\beta$) - $\beta \in V^*$ - set of characters that can be the first letter of a derivation starting from $\beta$\\
	First($\beta$) = \{a $|$ $\beta \implies *a\gamma$ \}\\
	
	So Predict(A, a) = \{A $\rightarrow \beta$ $|$ $\beta \implies$ *a$\gamma$ \}\\
	
	So really:\\
	
	Predict(A,a) = \{A $\rightarrow \beta$ $|$ a $\in$ First($\beta$) \} $\cup$ \{A $\rightarrow \beta$ $|$ Nullable($\beta$), a $\in$ Follow(A) \}\\
	Nullable($\beta$) = true if and only if $\beta \implies *\epsilon$\\
	Follow(A) = \{b $|$ S' $\implies$ *$\alpha Ab \beta$ \}\\
	- Terminal symbols that can come immediately after A in a derivation starting from S'\\
	
	\subsection*{Computing Nullable}
	\begin{verbatim}
	initialize Nullable[A] = false for all A
	repeat:
	-- For each rule B -> B1 ... Bk
	-- -- if k = 0 or Nullable[Bi] for all i
	-- -- -- Nullable[B] <- True
	until nothing changes
	\end{verbatim}
	
	\subsection*{Computing First}
	\begin{verbatim}
	initialize First[A] = {} for all A
	repeat
	-- for each rule B -> B1...Bk
	-- -- for i = 1 ... k
	-- -- -- if Bi is a terminal a
	-- -- -- -- First[B] += {a}
	-- -- -- -- break
	-- -- -- else
	-- -- -- -- First[B] += First[Bi]
	-- -- -- -- if not Nullable[Bi] break;
	until nothing changes
	\end{verbatim}
	
	\subsection*{Computing First*($\beta$)}
	- first set for a string of symbols\\
	\begin{verbatim}
	First*(Y1...,Yn)
	-- result <- empty
	-- for i = 1 ... n
	-- -- if Yi not in Sigma (ie non-terminal)
	-- -- -- result += First[Yi]
	-- -- -- if not Nullable[Yi] break;
	-- -- else (terminal)
	-- -- -- result += {Yi}
	-- -- -- break
	-- return result
	\end{verbatim}
	
	\subsection*{Computing Follow}
	\begin{verbatim}
	initialize Follow[A] = {} for all A != S'
	repeat
	-- for each rule B -> B1 ... Bn
	-- -- for i = 1....n-1
	-- -- -- if Bi is in N
	-- -- -- -- Follow[Bi] += First*(Bi+1 ... Bn)
	-- -- -- -- if all Bi+1 ... Bn are nullable (includes i = n)
	-- -- -- -- -- Follow[Bi] += Follow[B]
	until nothing changes
	\end{verbatim}
	
\end{document}
