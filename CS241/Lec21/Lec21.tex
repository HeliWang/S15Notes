\documentclass[12pt]{article}

\setlength\parindent{0pt}
\newcommand{\myt}[1]{\textbf{\underline{#1}}}

\usepackage{mathtools}
\usepackage{amssymb}

\title{\vspace{-15ex}CS241 Lecture 21\vspace{-1ex}}
\date{July 20th, 2015}
\author{Graham Cooper}

\begin{document}
	\maketitle
	
	\section*{Memory Allocation}
	- alloc.merl - must be linked LAST\\
	
	Add to the prologue:\\
	\begin{itemize}
		\item .import init
		\item .import new
		\item .import delete
	\end{itemize}
	
	\subsection*{Function init}
	\begin{itemize}
		\item sets up allocator's data structures
		\item call it once in the prologue
		\item takes a parameter in \$2\\
		- if calling with mips.array (ie. int wain(int * ...))\\
		- \$2 = length of the array\\
		- else \$2 = 0\\
	\end{itemize}
	
	\subsection*{New/Delete}
	\subsubsection*{New}
	new: \$1 = \# of words needed\\
	- returns ptr to memory in \$3\\
	- if allocation fails, return 0\\
	\begin{verbatim}
	code(new int [expr]) = 
	code(expr) ($3 <- expr)
	add $1, $3, $0
	call new
	bne $3, $0, $1
	add $3, $11, $0
	\end{verbatim}
	
	\subsubsection*{Delete}
	delete: \$1 = ptr to mem to be deallocated\\
	\begin{verbatim}
	code(delete [] expr) = 
	code(expr) ($3 <- expr)
	beq $3, $11, skipY
	add $1, $3, $0
	call delete
	skipY:
	\end{verbatim}
	
	You now have a compiler that handles pointers!!!\\
	
	\section*{Procedures}
	big picture:\\
	int f(...) \{...\}\\
	int g(...) \{...\}\\
	...\\
	int wain( , ) \{...\}\\
	
	Every function is going to have a prologue/epilogue\\
	
	\subsection*{Main Prologue/Epilog}
	\begin{itemize}
		\item save \$1, \$2 on stack
		\item import print, init, new, delete
		\item set \$4, \$11 etc if desired
		\item set \$29
		\item call init
		\item ... CODE
		\item reset stack and jr \$31
	\end{itemize}
	
	\subsection*{Procedure specific PRologues}
	\begin{itemize}
		\item don't need to set .imports, set constants
		\item set \$29!!!!
		\item save registers
		\item ... CODE
		\item restore registers
		\item reset stack ptr
		\item return
	\end{itemize}
	
	\subsection*{Saving and Restoring}
	\begin{itemize}
		\item procedure should save and store all regs that it will modify\\
		\item How do you know which regs?\\
		- If not sure, save and restore all of them\\
		- Our code geen scheme doesn't use any registers past 6, and we also use 29 - 31\\
		- If your scheme uses more regs, just keep track of them
		\item REMEMBER TO SAVE AND RESTORE REG 29
	\end{itemize}
	
	\subsubsection*{Two Approaches to Saving and restoring}
	caller-save cs. callee-save\\
	suppose f calls g\\
	
	callee-save: g saves all regs it modifies (this is what we are used to), then f doesn't worry about what g does\\
	
	caller-save: f saves all registers that contain critical data, do this before calling g, g does not have to worry as f has taken care of the saving\\
	
	Our approach has been: - caller-save for \$31, -callee-save for everything else\\
	
	Q: Who saves \$29? Callr or callee?
	
	IF \underline{callee-save}\\
	\begin{itemize}
		\item save \$29 along with other regs
		\item if you haven't yet set \$29, need to count back in the stack to find the beginning of the frame
		\item maybe easier to set \$29 first then save the regs, but we can't set \$29 before we save it!!
		\item so \underline{AT LEAST} save \$29 first, then set \$29, then save the other registers
		\item OR let the caller (f) save \$29 before calling g
	\end{itemize}
	\begin{verbatim}
	f(){ ... 
		...
		g()
		...
	}
	
	f:
	push($29)
	push($31)
	lis $5
	.word g
	jalr $5
	pop($31)
	pop($29)
	\end{verbatim}
	
	\section*{Labels}
	What if a program looks like:\\
	\begin{verbatim}
	int init(...){...}
	int print(...) {...}
	int else1(...) {...}
	\end{verbatim}
	
	procedure names match the names of existing labels.\\
	-won't assemble\\
	
	\subsection*{Solution:}
	\begin{itemize}
		\item make sure it can't happen
		\item use a naming scheme for labels that prevents duplication
		\item for functions, f, g, h, use the labels Ff, Fg, Fh, ie. reserve labels starting with F as denoted functions
		\item My compiler will not generate any labels that start with an F
	\end{itemize}
	
	\section*{Parameters}
	\begin{itemize}
		\item could use regs
		\item may not be enought
		\item or push params on stack
	\end{itemize}
	
	\begin{verbatim}
	factor ->ID(expr1, expr2, ...exprn)
	code(factor) = 
	push($29)
	push($31)
	code(expr1) ($3 <- expr1)
	push($3)
	...
	code(exprn)
	push($3)
	lis $5
	.word F__
	jalr $5
	POP ALL ARGUMENTS
	pop($31)
	pop($29)
	\end{verbatim}
	
	\begin{verbatim}
	procedure -> INT ID (Params) {dcls stmts return expr;}
	code(procedure) = 
	sub $29, $30, $4
	push regs
	code(dcls) (local vars)
	code(stmts)
	code(expr)
	pop regs
	add $30, $29, $4
	jr $31
	\end{verbatim}
	
	What does the stack look like?\\
	\begin{verbatim}
	int f(...){
	...
	g(...)
	...
	}
	\end{verbatim}
	
	\begin{tabular}{|c| c}
		& g's frame \\
		local vars & g's frame\\
		saved regs & g's frame \\ \hline
		args for g & f's frame \\
		\$31 & f's frame \\
		\$29 & f's frame \\
		& \\ \hline
		& wain;s frame \\ \hline
	\end{tabular}
	
	Suppose g is :\\
	\begin{verbatim}
	int g(int a, int b, int c){
	int d = 0; int e = 0; int f = 0;
	...
	}
	\end{verbatim}
	g's symbol table:\\
	\begin{tabular}{c | c | c }
		Name & Type & Offset \\ \hline
		a & int & 0\\
		b & int & -4\\
		c & int & -8\\
		d & int & -12\\
		e & int & -16\\
		f & int & -20\\
	\end{tabular}
	
	Params a,b,c below \$29\\
	Local vrs d,e,f above \$29\\
	In between are saved regs\\
	so symbol table offsets are wrong!\\
	
	Params should have positive offsets\\
	Local vars shoudl have negative offsets!\\
	
	Fix SymbolTable: Add 4 times number of args to all offsets in symboltable\\
	
	\begin{tabular}{c | c | c}
		Name & Type & Offset \\ \hline
		a & int & 12 \\
		b & int & 8 \\
		c & int & 4 \\
		d & int & 0 \\
		e & int & -4 \\
		f & int & -8 \\
	\end{tabular}
	
	d,e,f are still wrong, we hould push local variables first then save registers!!\\
	
\end{document}
