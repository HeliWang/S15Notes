\documentclass[12pt]{article}

\setlength\parindent{0pt}
\newcommand{\myt}[1]{\textbf{\underline{#1}}}

\usepackage{mathtools}
\usepackage{amssymb}

\title{\vspace{-15ex}CS 241 Lecture 12\vspace{-1ex}}
\date{June 15th, 2015}
\author{Graham Cooper}

\begin{document}
	\maketitle
	
	\myt{recall:} (see diagram)\\
	Input abab could be 1,2,3 or 4 tokens. \\
	Sol'n - take the $\epsilon$-move only when no other choice. ie. match the longest token possible\\
	Could fail, eg L = \{aa,aa\}, w = aaaa\\
	
	\section*{Concrete Realization: Maximal Munch Algorithm}
	Run DFA (no $\epsilon$-moves) until no non-error move is possible.\\
	If in an accepting state, token found\\
	else, back up to the most recent accepting state.\\
	-- -- input to that point is the new token, resume scanning from here\\
	end if\\
	Output token; $\epsilon$ move back to $q_0$\\
	
	We would use a variable to keep track of the most recent accepting state.\\
	
	\section*{Simplified Maximal Munch Algorithm}
	As above, but:\\
	If not inan accepting state when no transitions are possible, error, (ie. ignore backtracks)\\
	
	\myt{Example:}
	 Tokens: must start and end with a letter, can contain -\\
	 operator: --\\
	 Take: "ab--," as input\\
	 
	 once we scan to the second "-", no further move is possible but ab-- is not a valid token so we are not in an accepting state. 
	 Simplified MM: ERROR\\
	 MM: Back up to the previous accepting state (a,b) and then scan from there, we then have tokens (ab) and (--)\\
	 
	 MM Works, but Simplified MM does not.\\
	 
	 In practice, simplified MM is usually good enough.\\
	 
	 \myt{Example:} C++:\\
	 vector$<$vector$<$int$>>$ v;\\
	 C++ scans this as one token $>>$ rather than as two $>$'s\\
	 
	 \myt{C++}
	 Must seperate tokens by space.\\
	 Vector $<$vector$<$int$>$ $>$ v;\\
	
	\hrulefill\\
	
	What (if any) specific features of c (or scheme) programs cannot be verified with a DFA?\\
	
	Consider $\Sigma = \{(,)\}$\\
	$L = \{w \in \Sigma |$ w is a string with alanced parens\}.\\
	eg:\\
	$\epsilon \in L$\\
	$() \in L$\\
	$()() \in L$\\
	$(()) \in L$\\
	...
	$)( \notin L$\\
	
	Can we build a DFA for L?\\
	
	See picture\\
	
	Each new state recognizes one more level of nesting. But no finite number of states recognizes all levels of nesting. And DFA's have a finite number of states.\\
	
	\section*{Context Free Languages}
	Context free languages are languages that can be described by a context-free grammar.\\
	
	\subsection*{Intuition}
	Balanced parens.\\
	\begin{itemize}
		\item $S \rightarrow \epsilon$ "A word in the language is either empty"
		\item $S \rightarrow (S)$ or a word in the language surrounded by ( )
		\item $S \rightarrow SS$ or the concatentation of two words in the language
	\end{itemize}
	
	\myt{Shorthand:}\\
	$S \rightarrow \epsilon | (S) | SS$\\
	
	\myt{Show:} This system generates (())()\\
	$S \implies SS \implies (S)S \implies ((S))S \implies (())S \implies (())(S) \implies (())()$\\
	
	\myt{Notation:}\\
	"$\implies" \equiv $"derives"\\
	"$\alpha \implies \beta$" means $\beta$ can be obtained from $\alpha$ by \underline{one} application of a grammar rule\\
	
	\subsection*{Formal Definition}
	A context free grammar consists of:
	\begin{itemize}
		\item An alphabet $\Sigma$ of \underline{terminal symbols}
		\item A finite non-empty set N of \underline{Non-terminal symbols}
				N$\cap \Sigma = \emptyset$ (we use V ("vocabulary") to denote $N \cup \Sigma$)
		\item A finite set p of \underline{productions} which have the form: $A \rightarrow B$ where $A \in N, B in V^*$
		\item An alement S $\in$ N which is our start symbol
	\end{itemize}
	
	\subsection*{Conventions}
	a,b,c ... - elements of $\Sigma$ (characters)\\
	w,x,y,... - elements of $\Sigma^*$ (strings)\\
	A,B,C ... - elements of N (non-terminals)\\
	S - Usually the start symbol (not always)\\
	$\alpha, \beta, \gamma$... - elements of $V^*$ (ie $\Sigma \cup N)^*)$\\
	
	We write: $\alpha A \beta \implies \alpha \gamma \beta$ if there is a production A $\rightarrow \gamma$ in P. (RHS derivable from LHS in one step)\\
	
	$\alpha \implies *\beta$ - means $\alpha \implies ... \implies \beta$ (0 or more steps)\\
	
	\subsection*{Definitions}
	L(G) = \{$w \in \Sigma^* | S \implies *w \}$\\
	L(G) is the language specified by G, and $\Sigma^*$ are the strings of terminals derivable from S.\\
	
	A language L is \underline{Context-free} if L = L(G) for some context-free grammar G.\\
	
	\subsection*{Examples:}
	\subsubsection*{Palindromes over \{a,b,c\}}
	$S \rightarrow aSa | bSb | cSc | M$\\
	$M \rightarrow \epsilon | a | b | c$\\
	
	Show: $S \implies *abcba$\\
	$S \implies aSa \implies abSba \implies abMba \implies abcba$ (called a derivation)\\
	
	\subsubsection*{Expressions}
	$\Sigma = \{a,b,c,+,-,*,/\}$\\
	L = \{Arithmetic expressions using symbols from $\Sigma$ \}\\
	$S \rightarrow SOpS | a | b | c$\\
	$Op \rightarrow +|-|*|/$\\
	
	$\Sigma = \{a,b,c,+,-,*,/,(,)\}$
	$S \rightarrow SOpS | a | b | c | (S)$\\
	$Op \rightarrow + | - | * | /$\\
	
	Show: $S \implies *a+b$\\
	$S \implies SOpS \implies aOpS \implies a+S \implies a+b$\\
	We have a choice to expand which part, \\
	Leftmost derivation -  we should always expand the leftmost symbol first.\\
	Rightmost derivation - we always expand the rightmost symbol first\\
	
	
\end{document}
