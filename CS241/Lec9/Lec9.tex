\documentclass[12pt]{article}

\setlength\parindent{0pt}
\newcommand{\myt}[1]{\textbf{\underline{#1}}}

\usepackage{mathtools}
\usepackage{amssymb}
\usepackage{tikz,ifthen,amsmath,amssymb,fancyhdr,comment,lastpage}

\title{\vspace{-15ex}Lec 9 CS241\vspace{-1ex}}
\date{June 3rd, 2015}
\author{Graham Cooper}

\begin{document}
	\maketitle
	
	\section*{Regular Languages}
	built from:
	\begin{itemize}
		\item finite languages
		\item union
		\item concatenation
		\item repetition
	\end{itemize}
	- Union of two languages:\\
	$L_1 \cup L_2 = \{x | x \in L_1 or x \in L_2\}$\\
	- Concatenation of two languages:
	$L_1 \cdot L_2 = \{xy | x \in L_1, y \in L_2\}$\\
	eg: $L_1 = \{dog, cat\}$
	$L_2 = \{fish, \epsilon \}$
	$L_1L_2 = \{dogfish, dog, catfish, cat\}$\\
	
	- Repetition:
	$L^* = \{\epsilon \} \cup \{xy | x \in L^*, y \in L \}$\\
	$ = \{\epsilon \} \cup L \cup LL \cup LLL \cup .... $\\
	= 0 or more occurences of a word in L\\
	
	EG: L = \{a,b\}\\
	$L^* = \{\epsilon, a, b, aa, ab, bb, ba, aaa, aab, abb, ... \}$
	
	Show: $\{2^{2n}b | n \geq 0 \}$ is regular\\
	$(\{aa\})^* \cdot \{b\}$\\
	
	Shorthand - regular expression.\\
	
	\begin{tabular}{c | c | c}
		Language & Regular Expression & name \\ \hline
		\{\}     & $\emptyset$ & empty language\\
		$\{\epsilon \}$ & $\epsilon$ & language consisting of the empty word\\
		\{aaa\} & aaa & singleton language\\
		$L_1 \cup L_2$ & $L_1 | L_2$ & alternation (union) \\
		$L_1L_2$ & $L_1L_2$ & concatenation\\
		$L^*$ & $L*$ & repetition \\ 
	\end{tabular}
	ie: $\{a^{2n}b | n \geq 0\} = (aa)*b$\\
	
	\subsection*{Is C regular?}
	A C program is a sequence of tokens.\\
	
	A C program is a sequence of tokens, each of which comes from a regular language. C $\subseteq \{valid ctokens\}^*$ maybe.\\
	
	How can we recognize an arbitrary regular languae automatically?\\
	
	Eg. $\{a^{2n}b | n \geq 0 \} = (aa)^*b$\\
	Can we harness what we leanred about finite languages?\\
	- Yes if we allow loops in the diagram\\
	
	start $\overset{a}{\underset{a}{\rightleftarrows}}$ state\\
	$\underset{b}{\downarrow}$\\
	finished state\\
	
	\subsection*{Set of MIPS labels}
	start $\overset{a-z|A-Z}{\rightarrow}$ state(loops with a-z\textbar A-Z \textbar 0-9) $\overset{:}{\rightarrow}$ finished state\\
	
	\section*{Deterministic Finite Automata}
	These "machines" (state diagrams) are called Deterministic Finite Automata (DFAs)\\
	\begin{itemize}
		\item always start at the start state
		\item for each character in the input, follow the corresponding arc to the next state
		\item if on an accepting state when the input is exhuasted, you accept, else you reject.
	\end{itemize}
	
	What if there is no transition?
	start $\overset{a}{\underset{a}{\rightleftarrows}}$ state\\
	$\underset{b}{\downarrow}$\\
	finished state\\
	
	If you exit at the middle (not finished or start) state, you fall off the machine and are rejected.\\
	How to reject:
	\begin{itemize}
		\item you fall off of the machine
		\item you are not in an accepting state at end of input
	\end{itemize}
	
	More formally:\\
	There is an implicit "error state", all unlabeled transitions go to this error state.\\
	
	\myt{Example:} Strings over \{a,b\} wit han even number of a's and an odd number of b's\\
	
	$\rightarrow$ start state $\overset{b}{\rightarrow}$ accept state\\
	$\underset{a}{\downarrow} \underset{a}{\uparrow}$\\
	mid state $\overset{b}{\rightleftarrows}$ midstate (odd a, odd b) (this goes up to the accept state back and forth with a's)\\
	
	\subsection*{Formal Definition of a DFA}
	A DFA is a 5-tuple ($\Sigma, Q, q_0, A, \delta)$\\
	\begin{itemize}
		\item $\Sigma$ is a finite, non-empty set (alphabet)
		\item Q is a finite non-empty set (states)
		\item $q_0 \in Q$ (start state)
		\item A $\subseteq$ Q (accepting states)
		\item $\delta$ takes $q \times \Sigma \rightarrow Q$ (transition function: maps state + input character to next state)
	\end{itemize}
	
	$\delta$ consumes one character, can extend $\delta$ to a function that consumes an entire word:\\
	
	\myt{Definition: }\\
	$\delta^* (q,\epsilon) = q$\\
	$\delta^*(q, cw) = qc$\\
	
	We say a DFA($\Sigma, Q, q_0, A, \delta)$ accepts a word w if:\\
	$\delta^*(q_0, w) \in A$\\
	
	If M is a DFA, we denote by L(M) ("The language of M"), the set of all strings accepted by M\\
	L(M) = \{w $|$ M accepts w\}\\
	
	\subsection*{Theorem: Kleene}
	L is regular iff L = L(M) for some DFA M. (The regular languages are the languages accepted by DFA's.)\\
	
	
	
	
	
\end{document}
