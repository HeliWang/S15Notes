\documentclass[12pt]{article}

\setlength\parindent{0pt}
\newcommand{\myt}[1]{\textbf{\underline{#1}}}

\usepackage{mathtools}
\usepackage{amssymb}

\title{\vspace{-15ex}CS241 Lecture 11\vspace{-1ex}}
\date{June 10th, 2015}
\author{Graham Cooper}

\begin{document}
	\maketitle
	
	See notes for NFA\\
	
	\section*{NFA Trace: caba}
	
	\begin{tabular}{c | c | c}
		read & unread & states \\ \hline
		$\epsilon$ & caba & \{1\} \\
		c & aba & \{2,6\} \\
		ca & ba & \{3,5\} \\
		cab & a & \{4,5\} \\
		caba & $\epsilon$ & \{6\} \\
	\end{tabular}
	
	Build a DFA via the subset construction, see note\\
	
	\myt{Accepting:} any set that includes an accepting state from the original NFA.\\
	
	\myt{Obvious Fact:} Every DFA is implicitly an NFA.\\
	
	\myt{Also:} every NFA can be converted to a DFA for the same language\\
	
	\myt{So:} DFA's and NFA's accept the same class of languages\\
	
	\section*{$\epsilon$-NFA's}
	What if we let ourselves change states without reading a character?\\
	
	$\epsilon$-transitions: state $\overset{\epsilon}{\rightarrow}$ state\\
	- "free pass" to a new state without reading a character\\
	- makes it easy to glue smaller automata together\\
	
	See picture of $\epsilon$-NFA on notes\\
	
	\begin{tabular}{c | c | c}
		Read & Unread & states \\ \hline
		$\epsilon$ & caba & \{1,2,6\} \\
		c & aba & \{3,6\} \\
		ca & ba & \{4,7\} \\
		cab & a & \{5, 7\} \\
		caba & $\epsilon$ & \{6\} \\
	\end{tabular}
	
	$\therefore$ By the same renaming trick as before, every $\epsilon$-NFA has an equivalent DFA.\\
	$\therefore$ NFA's recognize the same class of languages as DFA's\\
	- The converstion can be automated\\
	
	If we can find an $\epsilon$-NFA for every regular expression, then we have one direction of kleene's theorem. (Regular expression $->$ $\epsilon$-NFA $->$ DFA)\\
	
	\section*{Regular Expression Types}
	\begin{enumerate}
		\item $\emptyset$ - $\epsilon$-NFA: non-accepting starting state.
		\item $\epsilon$ - $\epsilon$-NFA: Single accepting starting state.
		\item a - $\epsilon$-NFA: non-accepting starting state $\overset{a}{\rightarrow}$ accepting state
		\item $E_1|E_2$ - $\epsilon$-NFA: See paper
		\item $E_1E_2$ - $\epsilon$-NFA: See paper, E1's accepting states go to E2 and are no longer accepting.
		\item $E^*$ - $\epsilon$-NFA: See Paper
	\end{enumerate}
	
	$\therefore$ every regular expression has an equivalent $\epsilon$-NFA, and $\therefore$ an equivalent DFA. and the conversion can be automated ie. can write tools to convert regular expressions to DFA's.\\
	
	\section*{Scanning}
	Is C a regular language? Well,
	C keywords:
	\begin{itemize}
		\item ids
		\item literals
		\item operators
		\item commends
		\item punctiation
	\end{itemize}
	$\therefore$ sequences of these are also regular since the above are all regular\\
	
	So we can use finite automata to do scanning (tokenization). Ordinary DFA's answer yes/no to $w \in L$?\\
	We need: \\
	- Input string w\\
	- break into $w_1, w_2, ... w_n$ such that each $w_i \in L$ else error\\
	- output each $w_i$\\
	
	\myt{Consider:} L = \{valid c tokens\} is regular.
	Let $M_L = $ some machine, be the DFA that recognizes L\\
	Then, $M_L$ (with $\epsilon$ transitions going from end states to start state) recognizes $LL^*$\\
	
	Add an action to each $\epsilon$ move (See seperate paper)\\
	- The machine is now non-deterministic because $\epsilon$ moves are always optional.\\
	
	\myt{SO:} does this scheme represent a unique decomposition w = $w_1, w_2 ... w_n$?\\
	
	\myt{NO!} Consider the id portion: (See paper)\\
	input abab could be interpreted as 1,2,3 or 4 tokens\\
	
	What do we do about this?\\
	- Decide to take the $\epsilon$ move only if there is no other choice.\\
	$\implies$ always return the longest possible next token\\
	- could mean that valid matches are missed \\
	- consider L = \{aa, aaa\}, w = aaaa\\
	-- If you take the longest token first, aka aaa, then you will only be left with a which is not a token.\\
	-- The a left over cannot be matched :( poor a.\\
	
	
	
	
	
	
	
\end{document}
