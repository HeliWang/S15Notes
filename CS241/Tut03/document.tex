\documentclass[12pt]{article}

\setlength\parindent{0pt}
\newcommand{\myt}[1]{\textbf{\underline{#1}}}

\usepackage{mathtools}
\usepackage{amssymb}

\title{\vspace{-15ex}CS241 Tutorial 3\vspace{-1ex}}
\date{May 29th, 2015}
\author{Graham Cooper}

\begin{document}
	\maketitle
	\section*{Topics}
	\begin{itemize}
		\item Outputting bytes
		\item outputting isntructions
		\item MERL
		\item Relocating
	\end{itemize}
	
	\section*{Q1}
	Write psuedocode for a function output\_word which takes a 32\-bit integer and outputs each of its 4 bytes to standard output. Assume the existence of a function output\_byte which takes an int and outputs the low\-order byte if the integer can be represented in 8 bits\\
	
	\begin{verbatim}
	Void output_word(int word){
		char c;
		c = word >> 24;
		output_byte(c);
		c = word >> 16;
		output_byte(c);
		c = word >> 8l
		output_byte(c);
		c = word;
		output_byte(c);
	}
	\end{verbatim}
	
	\begin{verbatim}
	0110
	 >>1
	____
	0011
	
	0110
	 <<1
	____
	1100
	
	1100
	>>1
	____
	1110
	
	
	1100
	<<1
	____
	1000
	\end{verbatim}
	
	\section*{Q2}
	Write a function assemble\_add which assembles add instructions of the form add \$d, \$s, \$t given d, s and t and returns an int representing the binary encoding of the add instruction.
	
	
	\begin{verbatim}
	int assemble_add(int d, int s, int t){
		return  (s << 21) | (t < 16) | (d << 11) | 32;
	}
	\end{verbatim}
	
	\underline{MERL} Mips executible relactable linkable\\
	Stick notes:
	\begin{itemize}
		\item REL 0x1 loc
		\item ESD 0x5 val len
		\item ESR 0x11 loc len
	\end{itemize}
	
	
	\section*{Q3}
	\begin{verbatim}
	beq $0, $0, 2
	.word filelen
	.word codelen
	lis $6
	.word 0x18
	sw $31, -4($30)
	lis $31
	.word 4
	sub $30, $30, $31
	lis $3
	Rel1: 
	.word proc
	
	lis $11
	.word 1
	
	loop:
	beq $2, $0, end
	jalr $3
	sub $2, $2, $11
	beq $0, $0, loop
	
	end:
	add $3, $1, $0
	
	lis $31
	.word 4
	add $30, $30, $31
	lw $31, -4($30)
	jr $31
	proc:
	add $1, $1, $6
	jr $31
	
	codelen:
	.word 0x1
	.word Rel1
	
	
	file len: 
	\end{verbatim}
	
\end{document}
