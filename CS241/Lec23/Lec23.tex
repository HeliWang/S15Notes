\documentclass[12pt]{article}

\setlength\parindent{0pt}
\newcommand{\myt}[1]{\textbf{\underline{#1}}}

\usepackage{mathtools}
\usepackage{amssymb}

\title{\vspace{-15ex}CS 241 Lecture 23\vspace{-1ex}}
\date{July 27th, 2015}
\author{Graham Cooper}

\begin{document}
	\maketitle
	
	\section{Tail Recursion in WLP4}
	
	\begin{verbatim}
	int f(...) {
	-- if(...){
	-- -- if (..){	
	--	-- } else {}
	-- } 
	-- else {}
	return x;
	}
	\end{verbatim}
	
	Is the same as:\\
	
	\begin{verbatim}
	int f(...) {
	-- if(...){
	-- -- if (..){	
	--	-- } else {}
	-- return x;
	-- } 
	-- else {
	-- return x;
	--}
	}
	\end{verbatim}
	
	is the same as:\\
	\begin{verbatim}
	int f(...) {
	-- if(...){
	-- -- if (..){	
	-- -- return x; 
	-- -- } 
	-- -- else {
	-- -- return x;
	-- }
	-- return x;
	-- } 
	-- else {
	-- return x;
	--}
	}
	\end{verbatim}
	
	When return x follows an assignment to x, merge:\\
	x = f(...) $\rightarrow$ return f(...)\\
	return x;\\
	
	- may create some tail recursive calls\\
	
	Generalization:\\
	\begin{itemize}
		\item tail call optimization
		\item when a function's last action is any function call (recursive or not) can reuse the stack frame\\
	\end{itemize}
	
	\section*{Overloading}
	What would happen if we wanted to compile:\\
	
	\begin{verbatim}
	int f(int a){...}
	int f(int a, int *b){...}
	\end{verbatim}
	
	Get duplicated labels for f.\\
	How do we fix this?\\
	
	\subsection*{Name Mangling}
	Encode the types of params as part of the label\\
	
	Example naming convention:\\
	F + typeinfo + \_ + name\\
	ie.\\
	1. int f()\{...\}\\
	2. int f(int a)\{...\}\\
	3. int f(int a, int *b)\{...\}\\
	
	1. F\_f:\\
	2. Fi\_f:\\
	3. Fip\_f:\\
	
	
	\begin{itemize}
		\item C++ compilers wil ldo this because c++ has overloading
		\item there is no standard mangling convention
		\item all compilers are different
		\item makes it hard or  impossible to link code from different compilers
		\item this is by design b/c compilers differ in other aspects as well
	\end{itemize}
	
	C doesn't have overloading so there is no mangling\\
	- C and C++ code call each other routinely\\
	- How is this done?\\
	- Suppress mangling in c++\\
	Call C from C++\\
	- Extern "C" int f(int n); tells c++ f wot be mangled\\
	
	Call c++ from C - tell c++ not to mangle the function\\
	
	extern "C" int g(int x)\{...\} // dont mangle g\\
	
	and then obviously you cannot overload extern c functions\\
	
	\section*{Memory Management and the Heap}
	WLP4, C, C++\\
	\begin{itemize}
		\item explicit memory management
		\item user must free own data using free/delete
	\end{itemize}
	
	Java, Scheme\\
	\begin{itemize}
		\item implicit memory management
		\item garbage collection
	\end{itemize}
	
	\subsection*{How do new/delete or malloc/free work?}
	There are a variety of algorithms\\
	\subsubsection*{1)}
	List of free blocks:\\
	\begin{itemize}
		\item maintain a linked list of ptrs to blocks of free RAM
		\item Initially entire heap is free, list contains one entry
		\item Suppose heap is 1k
		\item suppose we allocate 16 bytes
		\begin{itemize}
			\item actually allocate 20 bytes + 1 int (4 bytes)
			\item return pointer to second word
			\item store size just before the returned pointer\
			\item free list ctonains the rest of the loop
		\end{itemize}
	\end{itemize}
	
	\myt{Note}: Repeated allocation and deallocation creates "holes" in the heap\\
	\myt{EG:}\\
	\begin{verbatim}
	alloc 20 {xx 20 xx,... (140)... ...}
	alloc 40 {xx 20 xx, xx 40 xx,...(100) ... ...}
	alloc 20 {...(20)..., xx 40 xx, ... (100)...}
	alloc 5 {xx 5 xx, ... (15). .., xx 40 xx, ... (100)...}
	etc.
	\end{verbatim}
	
	We get holes like the 15 block hole on the last line, this causes:\\
	\myt{Code fragmentation} - means even if n bytes are free, we may not be able to allocate n bytes\\
	
	To reduce fragmentation:\\
	- don't always pick the first block of RAM big enought to satisfy the request\\
	
	
	
\end{document}
