\documentclass[12pt]{article}

\setlength\parindent{0pt}
\newcommand{\myt}[1]{\textbf{\underline{#1}}}

\usepackage{mathtools}
\usepackage{amssymb}

\title{\vspace{-15ex}CS 241 Lecture 22\vspace{-1ex}}
\date{July 22, 2015}
\author{Graham Cooper}

\begin{document}
	\maketitle
	
	\section*{Procedure Register Problem}
	Recall: Save local vars first then save regs\\
	
	OR 
	
	Save all Regs in caller\\
	BUT:\\
	\begin{verbatim}
	f() {
	g();
	g();
	g();
	g();
	...
	...
	}
	
	g() {
	...
	}
	\end{verbatim}
	
	\begin{itemize}
		\item callee-save\\
		- saves regs once per function
		\item caller-save\\
		- saves regs once per call!\\
		- may cost you space as shown above\\
	\end{itemize}
	
	\section*{Boilerplate code for Function calls}
	\begin{verbatim}
	factor -> ID (expr1, exprn)
	code(factor) = 
	push($29)
	push($31)
	code(expr1)
	push($3)
	...
	code(exprn)
	push($3)
	lis $5
	.word FID
	jalr $5
	pop all args
	pop($31)
	pop($29)
	\end{verbatim}
	
	\begin{verbatim}
	code(procedure) = 
	sub $29, $30, $4
	push(dcls)
	push(reg)
	code(stmts)
	code(expr)
	pop(regs)
	add $30, $29, $4
	jr $31
	\end{verbatim}
	
	\section*{Optimization}
	Computationally unsolvable. Can only approximate.\\
	
	Eg 1+2\\
	\begin{verbatim}
	lis $3
	.word 1
	sw $3, -4($30)
	sub $30, $30, $4
	lis $3
	.word 2
	lw $5, 0($30)
	add $30, $30, $4
	add $3, $5, $3
	\end{verbatim}
	
	We could have it in 2 instructions:\\
	\begin{verbatim}
	lis $3
	.word 3
	\end{verbatim}
	- called constant folding
	
	\subsection*{Constant propagation}
	int x = 1;\\
	x + x;\\
	\begin{verbatim}
	lis $3
	.word 1
	sub $3, -4($30)
	sub $30m $30, $4
	lw $3, -12($29) (Assuming x is at -12($29))
	sw $3, -4($30)
	sub $30, $30, $4
	lw $3, -12($29)
	lw $5, 0($30)
	add $30, $30, $4
	add $3, $5, $3
	\end{verbatim}
	
	Could recognize that x = 1 and do:\\
	\begin{verbatim}
	lis $3
	.word 1
	sw $3, -4($30)
	sub $30, $30, $4
	lis $3
	.word 2
	\end{verbatim}
	
	IF that's the only place x is used, it doesn't need a stack entry. 
	\begin{verbatim}
	list $3
	.word 2
	\end{verbatim}
	
	Watch out for aliasing!!\\
	
	\begin{verbatim}
	int x = 1;
	int y = NULL;
	y = &x;
	*y = 2;
	x + x;
	\end{verbatim}
	
	Even if x's value is not known, could recognize that \$3 already contains x:\\
	\begin{verbatim}
	lw $3, -12($29)
	add $3, $3, $3
	\end{verbatim}
	This is known as common subexpression elimination\\
	- use a register to hold at b, then multiply it by itself, rather than compute a + b twice.\\
	
	\subsection*{Dead Code Elimination}
	\begin{itemize}
		\item if you are certain that some branch of a program will never run, do not output code for it. ie. if statements that are always false
	\end{itemize}
	
	\subsection*{Register Allocation}
	\begin{itemize}
		\item cheaper to use regs for vasr instead of the stack
		\item (saves lw's and sw's)
		\item eg, regs \$12-28 unused by our code gen
		\item we could use these registers to hold 17 variables and use the stack for the rest\\
	\end{itemize}
	Which 17 variables should we store on the stack?\\
	- probably the most used ones\\
	\myt{Issue:} \& address of operator
	- if you take the address of a variable, it can't be in a register, it must be in ram\\
	
	\subsection*{Strength Reduction}
	\begin{itemize}
		\item add usually runs faster than mult (prefer add over mult if we have a choice)
		\item could add to itself rather than multiply by 2
	\end{itemize}
	
	\subsection*{Procedure-specific Optimizations}
	\subsubsection{Inlining}
	\begin{verbatim}
	int f(int x){
	return x + x;
	}
	int wain(int a, int b){
	return f(a);
	}
	
	Could we turn wain into
	int wain(int a, int b){
	return a + a;
	}
	\end{verbatim}
	
	\begin{itemize}
		\item replace the function call wiht its body, right in the caller.
		\item saves the overhead of a function call
		\item IS it always a win?
		\item if F is called many times, we get many copies of its body\\
		\item some functions are harder to inline, ie recursive functions\\
		\item if all calls to f are inline, then we do not need to generate code for f\\
	\end{itemize}
	
	\subsubsection{Tail Recursion}
	\begin{verbatim}
	int fact(int n, int a){
	if (n == 0) return a;
	else return fact(n-1, n*a);
	}
	\end{verbatim}
	tail recursive because the recursive call is the last thigns that hte function does.\\
	No pending work to do when a recursive call returns\\
	Re-use the content of the current frame, so we do not need to pop the frame off the stack!\\
	Can this be done in WLP4?\\
	- no because return is at the end and no returns are inside of ifs
	
	but can we transform the program?
	
	\myt{Basic Transformation:}
	When return immediately follows an if-statement push inside both branches\\
	
	\begin{verbatim}
	if(){
	-- if(){
	-- }
	-- else {
	-- }
	}
	else {
	}
	return x;
	\end{verbatim}
	SAME AS
	\begin{verbatim}
	if(){
	-- if(){
	-- -- return x;
	-- }
	-- else {
	-- -- return x;
	-- }
	}
	else {
	-- return x;
	}
	\end{verbatim}	
	
	Whenever rturn x follows an assgn to x, merge:
	x = f(...)\\
	return x;\\
	
	into\\
	return f(...)\\
	
	\section*{WLP4?}
	
\end{document}
