\documentclass[12pt]{article}

\setlength\parindent{0pt}
\newcommand{\myt}[1]{\textbf{\underline{#1}}}
\newcommand*{\mipsPath}{../mips}%

\usepackage{mathtools}
\usepackage{amssymb}
\usepackage{\mipsPath}

\title{\vspace{-15ex}CS251: Intro to Computer Organization and Design\vspace{-1ex}}
\date{May 4th, 2015}
\author{Graham Cooper}

\begin{document}
	\maketitle
	
	\subsection*{Guiding Principals}
	\begin{itemize}
		\item Abstraction to simplify design
		\item Moores Law: Expect Rapid change in technology
		\begin{itemize}
			\item IC resources doubles every 18-24 months
			\item \# of transistors can fit on a circuit board will double
		\end{itemize}
		\item Improvement via Parallelism
		\item Improve Performance via Pipelining
		\item Improve Performance via Prediction
	\end{itemize}
	
	\subsection*{Big Picture:}
	\begin{itemize}
		\item Computer
		\item Control -> registers
		\item Datapath
		\item Processor
		\item Memory
		\item Input
		\item Output
	\end{itemize}{}
	
	\subsection*{Instruction Set Architecture}
	To connect to the hardware you must speak its lanuage (machine language/bytecode\\
	
	\subsection*{Basic MIPS}
	\$S1: f \$S2: g \$S3: h \$S4: i \$S5: j \$S6 = TEMP \$S7 = TEMP\\
	
	$f = (g + h) - (i + j)$
	
	
	
\end{document}
