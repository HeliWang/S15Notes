\documentclass[12pt]{article}

\setlength\parindent{0pt}
\newcommand{\myt}[1]{\textbf{\underline{#1}}}


\usepackage{mathtools}
\usepackage{amssymb}
\usepackage{listings}
\usepackage{mips}


\title{\vspace{-15ex}CS251: Intro to Computer Organization and Design\vspace{-1ex}}
\date{May 4th, 2015}
\author{Graham Cooper}

\begin{document}
	\maketitle
	
	\subsection*{Guiding Principals}
	\begin{itemize}
		\item Abstraction to simplify design
		\item Moores Law: Expect Rapid change in technology
		\begin{itemize}
			\item IC resources doubles every 18-24 months
			\item \# of transistors can fit on a circuit board will double
		\end{itemize}
		\item Improvement via Parallelism
		\item Improve Performance via Pipelining
		\item Improve Performance via Prediction
	\end{itemize}
	
	\subsection*{Big Picture:}
	\begin{itemize}
		\item Computer
		\item Control -> registers
		\item Datapath
		\item Processor
		\item Memory
		\item Input
		\item Output
	\end{itemize}{}
	
	\subsection*{Instruction Set Architecture}
	To connect to the hardware you must speak its lanuage (machine language/bytecode\\
	
	\subsection*{Basic MIPS}
	\$s1: f \$s2: g \$s3: h \$s4: i \$s5: j \$s6 = TEMP \$s7 = TEMP\\
	
	$f = (g + h) - (i + j)$\\
	\lstset{language=[mips]Assembler}
	\begin{lstlisting}
	add $s6, $s2, $s3
	add $s7, $s4, $s5
	sub $s1, $s6, $s7
	\end{lstlisting}
	
	\subsection*{Registers}
	Mips: \$0 does not write\\
	
	\subsection*{Instructions}
	\begin{itemize}
		\item R-format: results and registers
		\item I-format: use immediate values as well
		\item J-format: used for branching or jumps
	\end{itemize}
	
	\lstset{language=[mips]Assembler}
	\begin{lstlisting}
	beq $s1, $s2, 15
	\end{lstlisting}
	
	This will jump 15 ahead if \$s1 == \$s2, also I-format with the 15\\
	
	\myt{Program Counter (PC)} This is the line number you are on and it will always increase by 4 bytes for each instruction.
	
	
	
	
\end{document}
