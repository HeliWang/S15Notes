\documentclass[12pt]{article}

\setlength\parindent{0pt}
\newcommand{\myt}[1]{\textbf{\underline{#1}}}

\usepackage{mathtools}
\usepackage{amssymb}
\usepackage{listings}
\usepackage{mips}

\title{\vspace{-15ex}CS 251 - Lecture 2\vspace{-1ex}}
\date{May 4th, 2015}
\author{Graham Cooper}

\begin{document}
	\maketitle
	
	\myt{Binary Refresh:}\\
	$$101 = 1 \times 2^2 + 0 \times 2^1 + 1 \times 2^0$$
	
	\myt{Jump:}\\
	$28 \times 4 = 112$\\
	PC $\leftarrow$ offset $\times$ 4\\
	Overwrite PC.
	Example: j 28 sets the PC to 28 $\times$ 4\\
	
	\emph{jump and branch are I-format}
	\lstset{language=[mips]Assembler}
	\begin{lstlisting}
		beq \$1, \$2, 100
	\end{lstlisting}
	branch if equal\\
	if \$1 == \$2 go to offset relative to PC, else continue to next instruction\\
	
	We don't multiple by 4 so that we can get more space for the 16 bits we have for the value 100, ie going from 400, if we multiply by 4, it goes to 1600\\
	
	PC $\leftarrow 100 \times 4$ + (PC + 4)\\
	
	Slide 26:
	$$PC \Leftarrow (-3) \times 4 + (PC + 4)$$
	$$ = -12 + 120$$
	$$ = 108$$
	
	\myt{Memory Access - I-Format}\\
	\emph{jump and branch are I-format}
	\lstset{language=[mips]Assembler}
	\begin{lstlisting}
		lw $s1, 100($s2)
	\end{lstlisting}
	100(\$s2) is computing an address\\
	\$s1 $\Leftarrow$ M[address]\\
	\lstset{language=[mips]Assembler}
	\begin{lstlisting}
		sw $1, 100($2)
	\end{lstlisting}
	M[100 + \$s2] $\Leftarrow$ \$s1\
	\underline{Memory is written to!!}\\\\\\
	
	\myt{lw/sw Eample:}\\
	\lstset{language=[mips]Assembler}
	\begin{lstlisting}
		0 add $4, $7, $0
		4 addi $1, $0, 20
		8 lw $3 0($4)
		12 addi $1, $1, -1
		16 sw $3, O($5)
		20 addi $4, $4, 4
		24 addi $5, $5, 4
		28 bne $1, $0, -6
		32 add $8, $3, 0
	\end{lstlisting}
	0:4 - Setting \$4 to beginning of an array\\
	8:28 - Loop (goes 20 times)\\
	8 -  Load from memory into \$3\\
	16 - Store into Data memory, contents of \$3\\
	20:24 - Adding from one array to another array in memory\\
	
	\myt{ARM vs MIPS}\\
	add command has to add when using the \$0 register, while ARM has a move commmand which bypasses the add.\\
	
	
\end{document}
