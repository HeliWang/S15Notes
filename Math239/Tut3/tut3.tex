\documentclass[12pt]{article}

\setlength\parindent{0pt}
\newcommand{\myt}[1]{\textbf{\underline{#1}}}

\usepackage{mathtools}
\usepackage{amssymb}

\title{\vspace{-15ex}Math 239 Tutorial 3\vspace{-1ex}}
\date{May 26th, 2015}
\author{Graham Cooper}

\begin{document}
	\maketitle
	
	\section*{1)}
	\subsection*{a)}
	$$\frac{6-x+5x^2}{1-3x^2-3x^3} = \sum_{n=0}^{\infty}a_nx^n$$
	$$\iff 6-x+5x^2 = (1-3x^2-2x^3)\sum_{n=0}^{\infty}a_nx^n$$
	$$\iff 6-x+5x^2 = a_0 + a_qx + 6(a_2 - 3a_0)x^2 + \sum_{n = 3}{\infty}(a_n - 3a_{n-2} - 2a_{n-3})x^n$$
	
	\subsubsection*{Equating Coefficients}
	$a_0 = 6$\\
	$a_1 = -1$\\
	$a_2 - 3a_0 = 5$\\
	$a_2 = 5 + 18 = 23$\\
	$a_n - 3a_{n-2} - 2a_{n-3} = 0$\\
	
	Taking n = 5\\
	$a_5 = 3a_3 + 2a_2$\\
	$\implies a_5 = 9a_1 + 6a_6 + 2a_2 = 73$\\
	
	\subsection*{b)}
	Observe that $1-3x^2 - 2x^3 = (1-2x)(1+x)^2$\\
	
	And hence\\
	
	$$A(x) = \frac{6-x+5x^2}{(1-2x)(1+x)^2} = \frac{C_1}{1-2x} + \frac{C_2}{1-x} + \frac{C_3}{(1+x)^2}$$
	Find $C_1,C_2, C_3$ and use them to find an explicit formula for $a_n$\\
	
	Solution:\\
	$$\frac{6-x+5x^2}{(1-2x)(1+x)^2} = \frac{C_1}{1-2x} + \frac{C_2}{1-x}$$
	$$= \frac{C_1(1+x)^2 + C_2(1-2x)(1+x) + C_3(1-2x)}{(1-2x)(1+x^2)}$$
	after expanding
	$$\frac{(C_1 + C_2 + C_3) + (2C_1 - C_2 - 2C_3)x + (C_1 -2C_2)x^2}{(1-2x)(1+x)^2}$$
	Equating Coefficients:
	$$C_1 + C_2 + C_3 = 6$$
	$$2C_1 - C_2 - 2C_3 = -1$$
	$$C_1 - 2C_2 = 5$$
	$$C_1 = 3$$
	$$C_2 = -1$$
	$$C_3 = 4$$
	
	So
	$$A(x) = \frac{3}{1-2x} + \frac{-1}{1-x} + \frac{4}{(1+x)^2}$$\\
	$$= 3 \sum_{n=1}{\infty}2^nx^n - \sum_{n=0}^{\infty}(-1)^nx^n + 4 \sum_{n=0}^{\infty}(-1)^n(n+1)x^n$$
	$$= \sum_{n=0}^{\infty}[3 \cdot 2^n + (4n+3)(-1)^n]x^n$$
	$$a_n = 3 \cdot 2^n + (4n+3)(-1)^n$$
	
	Taking n = 5\\
	
	$$a_5 = 3 \cdot 32 + 23 - 1$$
	$$ = 73$$
	
	\section*{2)}
	
	By Binomial Theorem:\\
	$$[x^n](1-x)^k = (-1)^n{-k \choose n}$$
	$${-k \choose n} = \frac{(-k)(-k-1)...(-k-n+1)}{n!}$$
	$$= \frac{(-1)^nk(k+1)...(n+k-1)}{n!}$$
	$$= (-1)^k\frac{(n-k-1)!}{n!(k-1)!}$$
	$$= (-1)^n{n+k-1 \choose k-1}$$
	Then
	$$[x^n](1-x)^{-k}$$
	$$= (-1)^n{-k \choose n}$$
	$$= (-1)^n(-1)^n{n+k-1 \choose k-1}$$
	$$= (-1)^{2n}{n+k-1 \choose k-1}$$
	$$= {n+k-1 \choose k-1}$$
	
	\section*{3)}
	
	\subsection*{a)}
	Define $\alpha$: \{0,1\} $rightarrow$\\
	len $\rightarrow$ len + 1\\
	
	Observe that if $\sigma = \sigma_1 \sigma_2 ... \sigma_n \in S_n$ then\\
	w($\sigma$) = $\sum_{i=1}^{n}\alpha(\sigma_i)$\\
	
	Then $S_n$ = \{0,1\}$^n$ and so by the product lemma\\
	$$= \Phi_{S_n}(x) = \Pi_{i=1}^{n}\Phi_{\{0,1\}}(x)$$
	$$=(\Phi_{\{0,1\}}(x))^n$$
	$$=(x+x^2)^n$$
	
	\subsection*{b)}
	
	Observe that i) $S_i \cap S_j = \theta for i \neq j$\\
	ii) $T = \bigcap_{i=0}^{\infty}S_i$\\
	
	$\therefore$ by the sum lemma $\therefore \Phi_T(x) = \frac{1}{1-(x+x^2)}$\\
	$$\Phi_T(x) = \sum_{n=0}{\infty}\Phi_{S_n}(x)$$
	$$= \sum_{n=0}{\infty}(x+x^2)^n$$
	
	B/c $[x^0](x+x^2) = 0$\\
	We can convert it to the power series above
	
	\section*{4)}
	\subsection*{a)}
	
	We can think of S as \{0,...9\}$^6$ by appending leading zeros to "too short" integers."\\
	Define $\alpha = \{0...9\} \rightarrow N$ $x \rightarrow x$\\
	Then if $\sigma \in S$ we write $\sigma = \sigma_1\sigma_2...\sigma_6$ and note that w($\sigma) = \sum_{i = 1}^{6}\alpha(\sigma_i)$\\
	
	$\therefore$ by the product lemma\\
	$$\Phi_S(x) = \Pi^{6}\Phi_{\{0...9\}}(x)$$
	$$= (\Phi_{\{0...9\}}(x))^6$$
	It is easy to see that:\\
	$$\Phi_{\{0..9\}}(x) = 1 + x + x^2 + ... + x^9$$
	$$\Phi_S(x) = (1 + x + x^2 + ... x^9)^6$$
	$$= (\frac{1-x^{10}}{1-x})^6$$
	$\therefore$ the nuber of integers in S whose digits sum to k should be:\\
	$$[x^k](\frac{1-x^{10}}{1-x})^6$$
	
	\subsection*{b)}
	
	For each length l between 2 and 6 the string looks like this:\\
	$(\sigma, \sigma_2, ...\sigma_{l-1}, \sigma_{l+1})$\\
	For each l define $T_l$ to be the set of all strings of length l within the desired propery.\\
	Then:\\
	$$F_T(x) = (x^3 + x^5 + ... x^{17})(1+x + ...x^9)^{l-2}$$
	Then:\\
	$$\Phi_T(x) = \sum_{l=2}^{6}\Phi_{T_l}(x)$$
	$$ = \sum_{l=2}^{6}x^3(\frac{1-x^{10}}{1-x^2})(\frac{1-x^{10}}{1-x})^l$$
	The number we need is $[x^k]\Phi_T(x)$\\
	
	
	
	
\end{document}
