\documentclass[12pt]{article}

\setlength\parindent{0pt}
\newcommand{\myt}[1]{\textbf{\underline{#1}}}

\usepackage{mathtools}
\usepackage{amssymb}

\title{\vspace{-15ex}Math 239 - Lec 7\vspace{-1ex}}
\date{May 20th, 2015}
\author{Graham Cooper}

\begin{document}
	\maketitle
	
	\section*{Power Series}
	F(x) is a power series if $[x^0]g(x) = 0$\\
	If $[x^0]g(x) \neq 0$, f(g(x)) may or may not be a power series.
	
	$$f(x) = 1+x$$
	$$g(x) = 1+x$$
	$$f(g(x)) = f(1+x) = 1+1+x=2+x$$
	
	Let $A(x) = \sum_{n \geq 0}a_nx^n$ where $A(x) = \frac{1+2x}{1-5x+6x^2}$\\
	Multiply both sides by $1-5x+6^2$.
	$$(1-5x+6x^2)A(x) = 1+2x$$
	$$LHS = (1-5x+6x^2)(a_0 + a_1x + a_2x^2 + ...)$$
	$$= a_= + a_1x + a_2x^2 + a_3x^3 - 5a_0x - 5a_1x^2 - 5a_2x^3 - ...$$
	$$= a_0 + (a_1-5a_0)x + \sum_{n \geq 2}(a_n - 5a_{n-1} + 6a_{n-2})x^n$$
	
	This equals to 1+2x by comparing coeff:\\
	$$a_0 = 1$$
	$$a_1 - sa_0 = 2 \implies a_1 = 2+5 = 7$$
	$$a_n-5a_{n-1}+6a_{n-2} = 0$$
	(the last is for $n \geq 2$)\\
	$$a_n = 5a_{n-1} - 6a_{n-2}$$
	$$a_2 = 5a_1 - 6a_0 = 35-6=29$$
	$$a_3 = 5a_2 - 6a_1 = 103$$
	$$a_4 = 5a_3-6a_2 = 341$$
	
	$$A(x) = 1 + 7x + 29x^2 + 103x^3 + 341x^4 ... $$
	
	In general if $A(x) - \frac{P(x)}{Q(x)}$ where $Q(x) = 1 + q_1x + q_2x^2 + ... + q_kx^k$ then $a_n + q_1a_{n-1} + q_2a_{n-2} + ... + q_ka_{n-k}=0$ for $n \geq max(deg(P(X)) + 1, k)$\\
	
	\section*{Sum and Product Lemmas}
	
	Generating Series: Set S, weight function w.\\
	$$\Phi_s(x) = \sum_{\sigma \in S}x^{w(\sigma)} = \sum_{n \geq 0}a_nx^n$$
	$a_n$ = number of things in S of weight n.\\
	
	Sum Lemma: Let $S = A \cup B$ where $A \cap B = \emptyset$ (disjoint union)\\
	Let w be a weight function on S. Then:\\
	$\Phi_S(x) = \Phi_A(x) + \Phi_B(x)$\\
	
	Proof: 
	$$\Phi_S(x) = \sum_{\sigma \in S}x^{w(\sigma)}$$
	$$= \sum_{\sigma \in A}x^{w(\sigma)} + \sum_{\sigma \in B}x^{w(\sigma)}$$
	$$= \Phi_A(x) + \Phi_B(x)$$
	
	Example: $N_0 = \{0,1,2,3, ...\}$ Define w(a) = a\\
	$$\Phi_{N_0}(x) = 1 + x + x^2 + x^3 + ... = \frac{1}{1-x}$$
	Partition $N_0 = E \cup O$ where E is the set of all even integers in $N_0$, and O is the set of odd its.\\
	
	$$E = \{0,2,4,6...\}$$
	$$\Phi_E(x) = 1 + x^2 + x^4 + x^6 + ...= \frac{1}{1-x^2}$$
	$$O = \{1,3,5,7 ...\}$$
	$$\Phi_O(x) = x + x^3 + x^5 + x^7 + ... = \frac{x}{1-x^2}$$
	By Sum Lemma, 
	$$\Phi_E(x) + \Phi_O(x) = \frac{1}{1-x^2} + \frac{x}{1-x^2} = \frac{1+x}{1-x^2}$$
	$$= \frac{1+x}{(1+x)(1-x)} = \frac{1}{1-x} = \Phi_S(x)$$
	
	Product Lemma: Let A,B be sets with weight functions $\alpha$ and $\beta$ respectively\\
	Consider $A \times B$ with the weight function $w(a,b) = \alpha(a) + \beta(b)$\\
	Then
	$$\Phi_{A\times B}(x) = \Phi_A(x) \cdot \Phi_B(x)$$
	
	Example: $[6] \times [6]$ where $w(a,b) = a+b = \alpha(a) + \beta(b)$\\
	$$\Phi_{[6]}(x) = x + x^2 + x^3 + x^4 + x^5 + x^6$$
	By Product lemma,
	$$\Phi_{[6] \times [6]}(x) = (\Phi_{[6]})^2 = (x + x^2 + ... + x^6)^2$$
	$$=(x + x^2 + x^3 + x^4 + x^5 + x^6)(x + x^2 + x^3 + x^4 + x^5 + x^6)$$
	
	
	
	
\end{document}
