\documentclass[12pt]{article}

\setlength\parindent{0pt}
\newcommand{\myt}[1]{\textbf{\underline{#1}}}

\usepackage{mathtools}
\usepackage{amssymb}

\title{\vspace{-15ex}Math 239 - Lecture 3\vspace{-1ex}}
\date{May 8th, 2015}
\author{Graham Cooper}

\begin{document}
	\maketitle
	\section*{More on Bijections}
	S = subsets of [n]\\
	T = bin str of length n\\
	$f: T \rightarrow S $ bijection\\
	f(11010) = \{1,2,4\}\\
	
	f implies that $|S| = |T|$. $|T| = 2^n$ (n bits, 2 choices)\\
	So $|S| = 2^n$ each element of [n] is either in or out of the subset\\
	
	\subsection*{Proving Bijections}
	\myt{For this course you need:}\\
	\begin{itemize}
		\item clear definitions of $f: A \rightarrow B$
		\item show that $f(x) \in B$ for any $x \in A$
		\item Define the inverse $f^{-1}: B \rightarrow A$
	\end{itemize}
	
	\section*{Combinatorial Proofs}
	\subsection*{Binomial Theorem}
	$$(1+x)^n = \sum_{k = 0}^{n}{ n \choose k}x^k$$
	
	Prove by counting.\\
	$$(1+x)^n = (1+x)(1+x)...(1+x)$$
	Each term in the expansion is a product of n things, one from each bracket. $(1+x)^n$ is the sum of all such terms. Each term as the form $a_1 \cdot a_2 ... a_n$ where each $a_i$ is either 1 or x. This gives $x^k$ when k of the $a_i$'s are x's, nk of the $a_i$'s are 1's. There are ${n \choose k}$ ways to do so. So the coefficient of $x^k$ in $(1+x)^n$ is ${n \choose k}$. This proves the binomial theorem.\\
	
	\myt{Example:} Plug in x = 1 into the binomial theorem, we get:\\
	$$2^n = \sum_{k=0}^{n}{n \choose k}$$
	$$2^3 = {3 \choose 0} + {3 \choose 1} + {3 \choose 2} + {3 \choose 3}$$
	
	\myt{Combinatorial Proof:} Let S be all binary strings of length n.\\
	
	So $|S| = 2^n$ Let $S_k$ be the set of binary strings of length n with k 1's. Then $S = S_0 \cup S_1 \cup ... \cup S_n$ is a disjoint union. (each string has 0,1 r n 1's, and $S_i \cap S_j = \theta$ for $i \neq j$). We know $|S_k| = {n \choose k}$ (n bits, choose k to be 1s). So $|S| = |S_0| + |S_1| + ... + |S_n|$ and $2^n = {n \choose 0} + {n \choose 1} + ... + {n \choose n} = \sum_{k=0}^{n}{n \choose k}$.\\
	
	\myt{General:} Given S, count $|S|$ in 2 different ways. Since $|S|$ is fixed, the two ways are equal.\\
	(IMAGE OF PASCAL'S TRIANGLE)\\
	
	Identity: ${n \choose k} = {n-1 \choose k} + {n-1 \choose k-1}$ where $1 \leq k \leq n-1$\\
	
	Combinatorial Proof: Let S b the set of all subsets of $[n]$ of size k. Then $|S| = {n \choose k}$. Partition S into 2 sets $S_1, S_2$ where \\
	$S_1$ are subsets of $[n]$ of size k that include element n.\\
	$S_1$ are subsets of $[n]$ of size k that do not have element n.\\
	$$n = 5$$
	$$k = 3$$
	$$S = subsets of \{1,2,3,4,5\} of size 3$$
	$$S_1 = \{\{1,2,5\}, \{1,3,5\}, \{1,4,5\}, \{2,3,5\}, \{2,4,5\}, \{3,4,5\}\}$$
	$$S_2 = \{\{1,2,3\}, \{1,2,4\}, \{1,3,4\}, \{2,3,4\}\}$$
	
	Then $S = S_1 \cup S_2$ is a disjoint union, and $|S| = |S_1| + |S_2|$\\
	Each element of $S_1$ consists of n together with a subset of $[n-1]$ of size k -1\\
	So $|S_1| = {n-1 \choose k-1}$\\
	Each element of $S_2 = {n-1 \choose k}$\\
	$\implies {n \choose k} = {n-1 \choose k-1} + {n-1 \choose k}$\\
	
	
\end{document}
