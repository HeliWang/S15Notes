\documentclass[12pt]{article}

\setlength\parindent{0pt}
\newcommand{\myt}[1]{\textbf{\underline{#1}}}

\usepackage{mathtools}
\usepackage{amssymb}

\title{\vspace{-15ex}Math 239 - Tutorial 2\vspace{-1ex}}
\date{May 19th, 2015}
\author{Graham Cooper}

\begin{document}
	\maketitle
	\section*{Tutorial Problems}
	\subsection*{Problem 1}
	\subsubsection*{a)}
	Find $\Phi_{s_4}(x)$\\
	\begin{tabular}{c | c}
		$\sigma$ & $w(\sigma)$ \\
		0000 & 0 \\
		0001 & 0 \\ 
		0010 & 0 \\
		0011 & 1 \\
		0100 & 0\\
		0101 & 0\\
		0110 & 1\\
		0111 & 2\\
		1000 & 0\\
		1001 & 0\\
		1010 & 0\\
		1011 & 1\\
		1100 & 1\\
		1110 & 2\\
		1111 & 3\\
		1101 & 1\\
	\end{tabular}
	
	$\Phi_{s_4}(x) = \sum_{\delta \in s_4}x^{w^(\sigma)} = 8x^0 + 5x + 2x^2 + x^3$\\
	\subsubsection*{b)}
	Prove that for all $n \in N \Phi_{s_n}(x) = \Phi^*_{s_n}(x)$\\
	
	\myt{solution}: Define\\
	$$f: S_n \implies S_n$$
	$$x_1 \cdot x_2 ... x_n \implies (1-x_1)(1-x_2)...(1-x_n)$$
	Notice that f is a permutation of $S_n$\\
	Also notice that $w(x) = w^*(f(x)) \forall x \in S_n$\\
	$$\Phi_{S_n}(x) = \sum_{\sigma \in S_n}x^{w(\sigma)}$$
	$$= \sum_{\sigma \in S_n}x^{w^*(f(\sigma))}$$
	$$= \sum_{\sigma' \in S_n}x^{w^*(\sigma')}$$
	$$= \Phi_{S_n}^*(x)$$
	
	\subsection*{Problem 2}
	
	Notice that:
	$$f(x) = \frac{1}{1-x}$$
	$$g(x) = \frac{1}{1+x}$$
	Then 
	$$f(x)^2 = \frac{1}{(1-x)^2}$$
	$$= \sum_{i=0}^{\infty}{i+1 choose 1}x^i$$
	$$= \sum_{i=0}^{\infty}(i+1)x^i$$
	$$[x^{2015}]f(x)^2=2015+1 = 2016$$
	As well,
	$$f(x)g(x) = \frac{1}{(1-x)(1+x)}$$
	$$= \frac{1}{1-x^2}$$
	$$= \sum_{i=0}^{\infty}(x^2)^i = \sum_{i=0}^{\infty}x^{2i}$$
	$$[x^{2015}]f(x)g(x) = 0$$
	
	\subsection*{Problem 3}
	
	\myt{Solution:} By theorem of uniqueness of power series representation. We only need to prove equality of coefficients.\\
	
	$$[x^n]A(x)(B(x) + C(x))$$
	$$= \sum_{i=0}^{n}[x^i]A(x)[x^{n-i}](B(x) + C(x))$$
	$$= \sum_{i=0}^{n}[x^i]A(x)[[x^{n-i}]B(x) + [x^{n-i}]C(x)]$$
	$$= \sum_{i=0}^{n}([x^i]A(x)[x^{n-i}]B(x) + [x^i]A(x)[x^{n-i}]C(n))$$
	$$= \sum_{i=0}^{n}[x^i]A(x)[x^{n-i}]B(x) + \sum_{i=0}^{n}[x^i]A(x)[x^{n-i}]C(x)$$
	$$= [x^n]A(x)B(x) + [x^n]A(x)C(x)$$
	$$= [x^n](A(x)B(x) + A(x)C(x))$$
	
	\subsection*{Problem 4}
	\subsubsection*{a)}
	$$f(x) = \sum_{i=0}^{\infty}(-3x)^i - \sum_{n=142}^{\infty}(-3x)^i$$
	$$= \frac{1-3^{142}x^{142}}{1+3x}$$
	
	\subsubsection*{b)}
	$$h(x) = \sum_{i=1}{\infty}x^i$$
	Notice that $g(x) = h(\frac{x}{1-x^2})$. This power series composition is well-defined because $[x^0](\frac{x}{1-x^2}) = 0$
	
	$$g(x) = \sum{i=1}^{\infty}(\frac{x}{1-x^2})^i$$
	$$= (\frac{x}{1-x^2})\sum_{i=0}^{\infty}(\frac{x}{1-x^2})^i$$
	$$= (\frac{x}{1-x^2})(\frac{1}{1-\frac{x}{1-x^2}})$$
	$$= \frac{x}{1-x-x^2}$$
	
	\subsubsection*{c)}
	
	g(f(x)) is not defined because g has some power series representation but the constant term of g(x) is non-zero.
	
	\subsection*{Problem 5}
	$$\frac{1}{(1-x^3)^5} = \sum_{n=0}^{\infty}{n+4 \choose 4}x^{3n}$$
	$$\frac{1}{1-3x^2} = \sum_{m=0}^{\infty}3^mx^{2m}$$
	Where did the n go, why is there an n?\\
	
	We get $x^{11}$ from this product whenever we shoose m,n $\in$ N $\cup$ \{0\} such that $x^{2+3n+2m} = x^11$\\
	If and only if 2 + 3n + 2m = 11 if and only if 3n + 2m = 9\\
	The solutions are n=3, m=0, n=1, m=3\\
	$\therefore$ the coefficient of $x^11$ in $x^2(1-x^3)^{-5}(1-3x^2)^{-1}$ is ${7 \choose 4} + {5 \choose 4} \cdot 3^3$\\
	
	\subsection*{Alternate Problem 2 Solution}
	$$f(x)^2 = (1+x+x^2 + ...)(1+x+x^2+ ...)$$
	You'll get a contribution of 1 towarsd the coefficient of $x^{2015}$ for each solution to $i+j=2015$, $i,j \in N \cup\{0\}$.\\
	It's not hard to see that there are 2016 such pairs: \{(i, 2015-i)\} for i = 0 to 2015\\
	
	
	
\end{document}
