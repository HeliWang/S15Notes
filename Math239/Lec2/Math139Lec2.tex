\documentclass[12pt]{article}

\setlength\parindent{0pt}
\newcommand{\myt}[1]{\textbf{\underline{#1}}}

\usepackage{mathtools}
\usepackage{amssymb}

\title{\vspace{-15ex}Math 239 - Lecture 2\vspace{-1ex}}
\date{May 6th, 2015}
\author{Graham Cooper}

\begin{document}
	\maketitle
	\section*{Bijections}
	Let A, B be finite sets.\\
	Consider a function $f: \overset{domain}{A} \rightarrow \overset{codomain}{B}$\\
	
	\myt{One to One}\\\\
	f is 1 to 1 if $x \not \in A \implies f(x) \neq f(y) \in B$\\
	$(f(x) = f(y) \implies x = y )$\\
	
	%%IMAGE HERE 
	
	If f is 1 to 1 then $|B| \geq |A|$\\
	
	\myt{Onto}\\\\
	f is \underline{onto} if for all $y \in B$, there exists $x \in A$ such that $f(x) = y$ (Every element B is being mapped to by something in A).\\
	
	%% IMAGE HERE
	
	\myt{Bijection (One to One and Onto)}\\\\
	f is a bijection if f is one to one and onto.\\
	If f is a bijection, then $|A| = |B|$\\
	
	%% IMAGE HERE
	
	\myt{Example:} A = \{1, 2, 3\} B = \{a, b, c\} Define $f: A \rightarrow B$\\
	by: $f(1) = a$ $f(2) = b$, $f(3) = c$ f is a bijection\\
	
	\myt{Example:} Let S be the set of all subsets of [n] of size k.\\
	Let T be the set of all subsets of [n] of size n - k\\
	
	$$n = 4$$
	$$k = 1$$
	$$S = \{\{1\}, \{2\}, \{3\}, \{4\} \}$$
	$$T = \{\{1,2,3\},\{1,2,4\},\{1,3,4\},\{2,3,4\}\}$$
	$$\{1\} \rightarrow \{2,3,4\}$$
	$$\{2\} \rightarrow \{1,3,4\}$$
	$$\{3\} \rightarrow \{1,2,4\}$$
	$$\{4\} \rightarrow \{1,2,3\}$$\\
	
	The fractions below means that x is not in [n]\\
	
	Define $f: s \rightarrow T$ by $f(x) = \frac{[n]}{x}$, for any $x \in S$\\
	Check $f(x) \in T$. Since $x \leq [n]$ of size k, $\frac{[n]}{x}$ is also a subset of [n], now of size n-k. so $f(x) \in T$\\
	
	\myt{Inverse:}\\
	The inverse of $f: A \rightarrow B$ is the function $f^{-1}: B \rightarrow A$ such that for all $x \in A$, $f^{-1}(f(x)) = x$, and for all $y \in B$, $f(f^{-1}(y)) = y$\\
	
	\emph{Theorem: } $f: A \rightarrow B$ is a bijection if and only if its inverse exists\\
	
	Back to example, f has an inverse: $f^{-1}: T \rightarrow S$ where\\
	$f^{-1}(y) = \frac{[n]}{y}$ for all $y \in T$\\
	For any $x \in S$, $f^{-1}(f(x)) = f^{-1}(\frac{[n]}{x}) = \frac{[n]}{\frac{[n]}{x}}$\\
	
	This establishes that $|S| = |T|$ \\ $|S| = {n \choose k}$ $|T| = {n \choose n - k}$\\
	So ${n \choose k} = {n \choose n-k}$\\
	The bijection serves as a combinatorial proof of this equation.\\
	
	\myt{Example:}\\
	Let S be the set of all subsets of [n]\\
	Let T be the set of all binary strings of length n\\
	$$n = 3$$
	$$S = \{\phi, \{1\}, \{2\}, \{3\}, \{1, 2\}, \{1,3\}, \{2,3\}, \{1,2,3\}\}$$
	$$T = \{000, 001, 010, 011, 100, 101, 110, 111\}$$
	$$111 \rightarrow 1,2,3$$
	$$110 \rightarrow 1,2$$
	$$100 \rightarrow 1$$
	$$011 \rightarrow 2,3$$
	$$\downarrow$$\\
	Each 1 in the binary string is "on" for one of the digits 1, 2, and 3.\\
	
	Define $f: T \rightarrow S$ where $f(a_1, a_2, ...a_n) = $\{i | $i \in [n]$, $a_i = 1$\}\\
	(if $a_i$ is 1, put element i in the subset)\\\\
	
	The inverse is $f^{-1}: S \rightarrow T$ where for each $x \in S$\\
	f(x) = $a_1a_2...a_n$ hwere $a_i$ = \{1 if i $\in$ x | 0 if i not $\in$ x\}\\
	
	
	
	
	
	
	
	
\end{document}
