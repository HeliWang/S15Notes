\documentclass[12pt]{article}

\setlength\parindent{0pt}
\newcommand{\myt}[1]{\textbf{\underline{#1}}}

\usepackage{mathtools}
\usepackage{amssymb}
\usepackage{graphicx}

\title{\vspace{-15ex}Math 239 Lecture 31\vspace{-1ex}}
\date{July 22nd, 2015}
\author{Graham Cooper}

\begin{document}
	\maketitle
	\section*{Matchings}
	\myt{Definition:} An \underline{Alternating path} P with respect to a matching M is a path where consecutive edges alternate between being in M and not in M. An \underline{augmenting path} is an alternating path that starts and ends with unsaturated vertices.\\
	
	If there is an augmenting path, we can "switch" edges between the path and M to get a larger matching.\\
	
	\myt{Theorem} If there is an augmenting path with respect to M, then M is not a maximum matching.(By switching along the path, we saturate 2 more vertices, and get a matching that has one more edge.)
	
	\myt{Theorem} If there is no augmenting path with respect to M, then M is a maximum matching\\
	
	\section*{Vertex Cover}
	
	\myt{Defintition} A \underline{vertex cover} C of a graph G is a set of vertices such that each edge of G has at least one end in C.\\
	
	\myt{General Question:} Given a graph, what is the smallest size of a vertex cover?\\
	
	\myt{Theorem:} If M is any matching and C is any cover, then $|M| \leq |C|$.\\
	
	\myt{PRoof:} For each edge uv in M, at least one of u or v is in C. For different edges in M, different vertices are in C since matchings use distinct vertices So $|M| \leq |C|$\\
	
	\myt{Corollary:} If a mathing M and acover C satisfy $|M| = |C|$ then M is a max matching and C is a min cover\\
	
	
	
	
\end{document}
