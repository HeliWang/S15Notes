\documentclass[12pt]{article}

\setlength\parindent{0pt}
\newcommand{\myt}[1]{\textbf{\underline{#1}}}

\usepackage{mathtools}
\usepackage{amssymb}

\title{\vspace{-15ex}Math 239 Lecture 8\vspace{-1ex}}
\date{May 22th, 2015}
\author{Graham Cooper}

\begin{document}
	\maketitle
	
	\section*{Product Lemma}
	Recall:\\
	
	Sets A,B with weight $\alpha$, $\beta$\\
	Set A $\times$ B, with weight w(a,b) = $\alpha(a) + \beta(b)$\\
	Then $\Phi_{A\times B}(x) \cdot \Phi_B(x)$\\
	
	\subsection*{Proof of the Product Lemma}
	$$\Phi_A(x) \cdot \Phi_B(x) = (\sum_{a \in A}x^{\alpha(a)})(\sum_{b \in B}x^{\beta(b)})$$
	$$= \sum_{a \in A}\sum_{b \in B}x^{\alpha(a)}x^{\beta(b)}$$
	$$\sum_{(a,b) \in A \times B}x^{\alpha(a) + \beta(b)}$$
	$$= \sum_{(a,b) \in A \times B}x^{w(a,b)}$$
	$$= \Phi_{A\times B}(x)$$
	
	\myt{Example:} Let $N_0 = \{0,1,2,3,...\}$ w(a) = a. Then:\\
	
	$$\Phi_{N_0}(x) = 1 + x + x^2 + x^3 + ... = \frac{1}{1-x}$$
	$\frac{1}{(1-x)^k}$ is the generating series for $N_0 \times N_0 ... \times N_0 = N_0^k$\\
	Where $w(a_1,a_2...a_k) = a_1 + a_2 + ... + a_k$ by product lemma.\\
	
	So $[x^n]\frac{1}{(1-x)^k}$ is the number of k tuples ($a_1...a_k) \in N_0^k$ where they sum to n.\\
	$\iff$ the number of non-negative integer solutions to $a_1 + a_2 + ,,, + a_k = n$\\
	
	
	
\end{document}
