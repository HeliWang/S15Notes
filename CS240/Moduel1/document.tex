\documentclass[12pt]{article}

\setlength\parindent{0pt}
\newcommand{\myt}[1]{\textbf{\underline{#1}}}

\usepackage{mathtools}
\usepackage{amssymb}

\title{\vspace{-15ex}CS 240 Module 1: Introduction and Asymptotic Analysis\vspace{-1ex}}
\date{May 5th, 2015}
\author{Graham Cooper}

\begin{document}
	\maketitle
	\section*{What is this course about?}
	\myt{Data Structures}
	\begin{itemize}
		\item hand in hand with algorithms
		\item patterns for storing/maintaining large data
		\item perform operations on the data (insertion, deletion, sorting)
		\item algorithms for these operations
	\end{itemize}
	
	\myt{Example, Student Records}
	\begin{itemize}
		\item Abstract Data Type $\rightarrow$ a concept (eg. Stack)
		\item Data Structure $\rightarrow$ implementation of a data type (eg, Linked List)
	\end{itemize}
	
	Algorithms are presented using pseudocode and analyzed using order notation
	
	\subsection*{Dictionary ADT}
	\begin{itemize}
		\item a set of items with the following operations
		\item insertion, deletion, search
	\end{itemize}	
	
	\subsection*{Problems}
	\subsubsection*{Problem of Sorting}
	Given a $\overset{input}{set/array}$ of $\overset{size}{size n}$ of numbers, 
	$\overset{operation(output)}{arrange them in increasing order}$
	
	\subsubsection*{Solution for a problem - Algorithm}
	Different solutions/algorithms eg. bubble sort/quicksort to solve sorting problems
	
	The Algorithm should be correct and efficient(Time complexity and space).
	
	\underline{RAM Random Access Memory:} No matter where you are accessing memory, it takes the same about of time
	
	This can also be thoguht of as a random access machine which is an abstract machine
	
	$$f(n) \in O(g(n)) \overset{informal}{=} f(n) \leq g(n)$$
	$$f(n) \in \Omega(g(n) \overset{informal}{=} f(n) \geq g(n)$$
	$$f(n) \in \Theta(g(n)) \overset{informal}{=} f(n) = g(n)$$
	$$f(n) \in o(g(n)) \overset{informal}{=} f(n) < g(n)$$
	$$f(n) \in \omega(g(n)) \overset{informal}{=} f(n) > g(n)$$
	
	Number of primitives plus r.a.o. time complexity
	
	
	$$f(n) = 100n + 200$$
	$$g(n) = 1/2n^2$$
	$$Show f(n) \in O(g(n))$$
	$$\exists c, 100n + 200 \leq c \cdot 1/2n^2$$
	$$n \geq n_0$$
	$$Lets have c = 1/2$$
	$$100n + 200 \leq 1/2n^2 \cdot 1.2$$
	$$100/n + 200/n^2 \leq 1/2 \cdot 1/2$$
	$$if n > 200$$
	$$100/n + 200/n^2 < 100/200 + 200/200^2 \leq |1/2 + 1/2| = 1 \cdot 1/2$$
	
	\section*{Order Notation} - May 7th
	
	$f(n) \leftarrow O(g(n)) \iff \exists c,n_0$ $\forall n > n_0$ $0 < f(n) < cg(n)$\\
	$f(n) = 2n^2 + 3n + 11$ $g(n) = n^2$\\
	prove $f(n) \in O(g(n))$\\
	We need to find c,$n_0$ such that $\forall n > n_0$
	$$2n^2 + 3n + 11 < c \cdot n^2$$
	$$\iff 2 + \frac{3}{n} + \frac{11}{n^2} < c$$
	$$n_0 = 1 \implies c > 2 + 3 + 11 = 16$$
	$$n_0 = 2 \implies c > 2 + 3/2 + 11/4$$
	$\therefore n_0 = 1$ and $c = 16$
	
	$$f(n) = 2010n + 1388n$$
	$$g(n) = n^3$$
	
	prove $f(n) \in o(g(n))$
	
	find $n_0 > 0$, such that for all $c > 0$ $2010n^2 + 1388n < cn^3$
	
	I can express n as a function of c
	
	$$cn^3 - 2010n^2 - 1388n > 0$$
	$$cn^2 - 2010n - 1388 > 0$$
	$$\delta = 2010^2 + 4*c*1388$$
	$$n > n_0 = \frac{2010 +\sqrt{2010^2 + 4*c*1388}}{4}$$
	[IMAGE 2]
	
	show $f(n) \in \Theta(g(n))$
	
	$f(n) = n + 2\sqrt{n}log(n)$ $g(n) = n$
	
	$log(n) \in o(\sqrt{n})$
	
	$\sqrt{n}log(n) \in o(\sqrt{n} \times \sqrt{n}) = o(n)$
	
	We need to find $c_1$ $c_2$, $n_0$ such that
	
	$$c_1 \times n \leq n + 2 \sqrt{n}logn(n) \leq c_2 \times n$$
	$\forall n > n_0$
	
	$c_1 = 1$
	
	for finding $c_2$
	
	$$n + 2\sqrt{n}log(n) \leq c_2 \times n$$
	$$1 + \frac{2\sqrt{n} \times log(n)}{n} \leq c_2$$
	$$n_0 = 64$$
	$$c_2 \geq 1 + \frac{2\sqrt{64}log(64)}{64} = 1 + \frac{2 \times 8 \times 6}{64} > 3$$
	$$c_2 = 3 works$$
	$c_1 = 1$ $c_2 = 3$ $n_0 = 64$
	
	Could use any value not just 64
	
	
	\begin{tabular}{c | c}
		Algebraic & Asymptotic \\ \hline
		$f(n) = g(n)$ & $f(n) \in O(g(n))$ \\ \hline
		$f(n) < g(n)$ & $f(n) \in o(g(n))$ \\ \hline
		$f(n) > g(n)$ & $f(n) \in \omega(g(n))$ \\ \hline
		$f(n) \leq g(n)$ & $f(n) \in O(g(n))$ \\ \hline
		$f(n) \geq g(n)$ & $f(n) \in Theta(g(n))$
	\end{tabular}
	
	$$f(n) \in \theta(1)$$
	$$f(n) \in \theta(log(n))$$
	$$f(n) \in \theta(log^kn)$$
	$$f(n) \in \theta(\sqrt{n})$$
	$$f(n) \in \theta(n)$$
	
	\myt{example, Slide 33}\\
	$f(n) = log(n)$ $g(n) = n^i$
	$$\lim_{n/to/infty}\frac{f(n)}{g(n)} = \lim_{n/to/infty}\frac{log(n)}{n^i}$$
	$$ (derivitive) = \frac{c \cdot \frac{1}{n}}{i\times n^{i-1}} = \frac{c}{i \times n^i} \implies 0$$
		$$ \implies f(n) \in o(g(n))$$
		
		\subsection*{Order Notation} - May 12th
		
		(slide 33)\\
		$$\lim_{n \rightarrow \infty } \frac{logn}{n^i}$$
		$$\lim_{n \rightarrow \infty} \frac{\frac{1}{n}}{in^{i-1}}$$
		$$log(n) \in o(n^i)$$
		
		(Slide 34)\\
		
		$$\forall n > n_0$$
		$$n < \underset{[1,3]}{(2 + sin(n7/2))} \leq 3n$$
		$$c_1g(n) \leq f(n) \leq c_2g(n)$$
		$$g(n) = n$$
		
		(Slide 36)\\
		
		$$f(n) \in O(g(n))$$
		$$\implies \exists c, n_0, \forall n > n_0, f(n) \leq c_1g(n)$$
		
		$$h(n) \in O(f(n) + g(n))$$
		$$\exists c,n_0, \forall n > n_0$$
		$$h(n) \leq c(f(n) + g(n)) \leq c(2\cdot max\{f(n),g(n)\} \leq 2c(max\cdot\{f(n),g(n)\})$$
		
		$$f(n) = n^2 + \sqrt{n}log^{1000}n + n$$
		$$f(n) \in O(max \{n^2, \sqrt{n}log^{1000}n + n\})$$
		$$f(n) \in O(n^2)$$
		$$[formally prove \sqrt{n}log^{1000}n \in o(n^2)]$$
		
		$$\forall n > n_0 f(n) < c_1g(n)$$
		$$\exists c_1, c_2, n_0 g(n) < c_2h(n) \rightarrow f(n) < c_1 c_2 h(n)$$
		
		(slide 37)\\
		
		\myt{Arithmetic Sequence}
		
		$$\sum_{i=0}^{n-1}(a+di) = \sum_{i=0}^{n-1}a + \sum_{i=0}^{n-1}(di)$$
		$$= an + d\sum_{i=0}^{n-1}i$$
		$$= an + d\frac{n(n-1)}{2} \in \Theta(n^2)$$
		$$0 + 1 + 2 + ... + (n-1) = \frac{n(n-1)}{2}$$
		
		k=2$$1 + 4 + 9... + n^2 = \frac{n(n+1)(2n + 1)}{6} \in \Theta(n^3)$$
		k=3$$1 + 8 + 27...+ n^3 = (\frac{n(n+1)}{2})^2 \in \Theta(n^4)$$
		k=4$$1 + 16 + 82...+n^4 = \Theta(n^5)$$
		
		k=x$$\sum_{i=1}^{n}(a+di^k) \in \Theta(n^{k+1})$$
		
		\myt{Geometric Sequence}
		
		$$\sum_{i=1}^{n}ar^i \in \Theta(r^n) for r > 1$$
		$$\in \Theta(n) for r = 1$$
		$$\in \Theta(1) for r < 1$$
		
		$$r=2:a(1+2+4+8...+2^n) \in \Theta(2^n)$$
		$$r=1/2: a(1+1/2+1/4+1/8 ...) < 2a \in \Theta(1)$$
		
		\myt{Harmonic Sequence}
		
		$$\sum_{i=1}{n}\frac{1}{i} \in \Theta(logn)$$
		$$ln(n+1) < \sum_{i=1}^{n}log(i) < ln(n) + 1$$
		$$ln(n) = \frac{log_2n}{log_cn} \in \Theta (logn)$$
		
		\myt{Misc Math Facts}\\
		(slide 38)\\
		
		$$\sum_{i=1}^{n}ir^i \in \Theta(nr^n)$$
		$$1+1/2+1/4+1/8 ... + 1/n <\approx 2$$
		$$1+1/4+1/9+1/16+...+1/n^2 <\approx \frac{\pi^2}{6}$$
		
		$$n! \in o(n^n)$$
		$$logn! \in log(n^n) = \Theta(nlogn)$$
		
		\subsection*{Loop Analysis}
		slide 41\\
		
		$$T(n) = c_1 + \sum_{i=1}^{n}\sum_{j=i}^{n}c_2$$
		$$=c_1 + \sum_{i=1}^{n}(n-i+1)c_2$$
		$$=c_1 + c_2 \sum_{k=1}^{n}k$$
		$$=c_1 + c_2 \times \frac{n(n+1)}{2} \in \Theta(n^2)$$
		
		SLide 42\\
		
		$$T(n) = c_1 + \sum_{i=1}^{n} \sum_{j=i}^{n}(c_2 + \sum_{k=i}^{j}c_3)$$
		$\sum_{k=i}^{j}c_3 = (j-i+1) \times c$ aside*
		$$ = c_1 + \sum_{i=1}{n}\underset{A}{\sum_{j=i}^{n}(c_2 + (j-i+1)c_3)}$$
		$A = \sum_{j=i}^{n}c_2 + c_3 \sum_{j=i}^{n}(j-i + 1)$ Aside* let $t=j-i+1$\\
		$A = (n-i+1)c_2 + c_3 \sum_{t=1}^{n-i+1}t$\\
		$A=(n-i+1)c_2 + c_3 \frac{(n-i+1)(n-i+2)}{2}$
		$$ = c_1 + \sum_{i=1}^{n}((n-i+1)c_2 + c_3 + \frac{(n-i+1)(n-i+2)}{2})$$
		Let L = n-i+1
		$$= c_1 + \sum_{L=1}^{n}(Lc_2 + c_3\frac{(L)(L+1)}{2})$$
		$$= c_1 + \sum_{L=1}^{n}(\frac{c_3}{2}L^2 + (c_2 + \frac{c_4}{2})L)$$
		$$= c_1 \frac{c_3}{2} \times \frac{n(n+1)(2n+1)}{6} + (c_2 + \frac{c_3}{2}\frac{n(n+1)}{2}$$
		$$= \frac{1}{6} \times \frac{c_3}{2}n^3 + o(n^3) \in \Theta(n^3)$$
		
		Slide 43
		
		$$T(A_4) = c_1 + \sum_{i=1}^{n}(c_2 + \log{i})$$
		$$ = c_1 + c_2 \times n + \sum_{i=1}{n}\log{i}$$
		Aside: $\log{n!} \in \Theta(nlogn)$
		$$= \Theta(n\log{n}) + o(n\log{n}) \in \Theta(n\log{n})$$
		
		\subsection*{MergeSort}
		
		5 2 1 4 \textbar 6 8 0 3\\
		1 2 4 5 \textbar 0 3 6 8\\
		0 1 2 3 4 5 6 8\\
		
		\subsubsection*{Analysis of Mergesort}
		T(n) = 
		$\Theta(1)$ if n = 1\\
		$T(ceil(\frac{n}{2})) + T(floor(\frac{n}{2})) + \Theta(n)$ for n $>$ 1\\
		
		If n is a power of 2
		
		$$T(ceil(\frac{n}{2})) + T(floor(\frac{n}{2})) + \Theta(n)$$
		$$= 2T(\frac{n}{2}) + cn$$
		$$= 4T(\frac{n}{4}) + 2 \times c\frac{n}{2} + cn$$
		$$= 8T(\frac{n}{8}) + 4 \times c\frac{n}{4} + 2\times c\frac{n}{2} + cn$$
		After logn strps
		$$= 2^{logn} \times T(1) + cn + cn +... + cn$$
		$$ = n\times d _ cn + \log{n}$$
		$$ \in \Theta(n\log{n})$$
	
	
\end{document}
