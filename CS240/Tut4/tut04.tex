\documentclass[12pt]{article}

\setlength\parindent{0pt}
\newcommand{\myt}[1]{\textbf{\underline{#1}}}

\usepackage{mathtools}
\usepackage{amssymb}

\title{\vspace{-15ex}CS240 - Tutorial 4\vspace{-1ex}}
\date{May 27th, 2015}
\author{Graham Cooper}

\begin{document}
	\maketitle
	
	\section*{QuickSelect Example}
	
	\subsection*{Q1}
	Find the 3rd smallest element in A\\
	Pivot = A[0]\\
	A = \\
	\begin{tabular}{|c | c | c | c | c | c | c | c | c | c|}
		\hline
		8 & 17 & 10 & 1 & 6 & 20 & 2 & 9 & 7 & 13 \\ \hline
	\end{tabular}\\
	\begin{tabular}{|c | c | c | c | c | c | c | c | c | c|}
		\hline
		\underline{8} & \underline{17} & 10 & 1 & 6 & 20 & 2 & 9 & \underline{7} & 13 \\ \hline
	\end{tabular}\\
	\begin{tabular}{|c | c | c | c | c | c | c | c | c | c|}
		\hline
		8 & 7 & 10 & 1 & 6 & 20 & 2 & 9 & 17 & 13 \\ \hline
	\end{tabular}\\
	\begin{tabular}{|c | c | c | c | c | c | c | c | c | c|}
		\hline
		8 & 17 & \underline{10} & 1 & 6 & 20 & \underline{2} & 9 & 7 & 13 \\ \hline
	\end{tabular} \\
	\begin{tabular}{|c | c | c | c | c | c | c | c | c | c|}
		\hline
		8 & 7 & 2 & 1 & 6 & 20 & 10 & 9 & 17 & 13 \\ \hline
	\end{tabular}\\
	Place Pivot\\
	\begin{tabular}{|c | c | c | c | c | c | c | c | c | c|}
		\hline
		6 & 7 & 2 & 1 & 8 & 20 & 10 & 9 & 17 & 13 \\ \hline
	\end{tabular}\\
	
	Recurse on the left hand side (6,7,2,1)\\
	pivot = A[0]\\
	\begin{tabular}{|c | c | c | c|}
		\hline
		6 & \underline{7} & 2 & \underline{1} \\ \hline
	\end{tabular}\\
	\begin{tabular}{|c | c | c | c|}
		\hline
		6 & 1 & 2 & 7 \\ \hline
	\end{tabular}
	\begin{tabular}{|c | c | c | c|}
		\hline
		2 & 1 & 6 & 7 \\ \hline
	\end{tabular}
	
	Recurse on left hand side (2, 1)\\
	
	Since the pivot is at index 3, it correspons to the 3rd smallest element, we are done.\\
	
	\subsection*{Q2)}
	Assume A has distinct elements.\\
	\begin{verbatim}
	Bogo(A){
		shuffle(A) //O(n)
		if A is sorted{
			return A;
		}
		else {
			return Bogo(A);
		}
	}
	\end{verbatim}
	
	Best case is O(n)\\
	Worst case is O($\infty$)\\
	May not terminate\\
	
	$$T_{avg}(n) = 1 \cdot cn + \underset{A is sorted}{(\frac{1}{n!} \cdot d)} + \underset{A is not sorted}{((1-\frac{1}{n!}))T_avg(n)}$$
	$$T_{avg}(n)[1-(1-\frac{1}{n!})] = cn + \frac{1}{n!} \cdot d$$
	$$T_{avg}(n) = cn \cdot n! + d \in O(n \cdot n!)$$
	
	EE[x] = $\sum_{x \in X}P_r(x) \cdot RunningTime(X)$\\
	 
	\subsection*{Q3)}
	
	Toss identical balls at random into buckets (or bins), one at a time, uniformly at random. How many tosses can we expect to make such that every bucket contains at least 1 ball.\\
	
	- Define a toss in which a ball falls into an empty bucket as a hit and a non-empty bucket as a miss\\
	- Partition the tosses into stages\\
	-- The $i^{th}$ stage consists of the tosses after the $(i-1)^{th}$ hit until (And including) the $i^{th}$ hit.\\
	
	Ex. 4 Buckets\\
	Toss sequence: 2$|$,2,3$|$,4$|$,3,3,2,4,1$|$\\
	
	During the $i^{th}$ stage.\\
	- (i-1) non-empty buckets \\
	- (b - i + 1) empty buckets (b is the bumber of buckets) \\
	- Pr(throwing in empty bucket) = $\frac{b-i+1}{b}$\\
	
	Define $n_i$ = number of throws in stage i\\
	$$n = \sum_{i = 1}^{b}n_i$$
	$$EE[n] = EE[\sum_{i=1}^{b}n_i]$$
	$$= \sum_{i=1}{b}EE[n_i]$$
	the above by linearity of expectation\\
	$n_i \approx$ geometric dist.\\
	EE[$n_i]$ = 1/p\\
	$$= b\sum_{i=1}^{b} \frac{1}{b-i+1}$$
	$$= b\sum_{i=1}^{b}\frac{1}{i} \in \Theta(blnb)$$
	
	\subsection*{Q4}
	Argue that any comparison based sorting algorithm requires at least 7 comparisons to S numbers\\
	
	- An algorithm performs actions as the result of different comparisons.\\
	
	If (A[i] $<$ A[j]) then do stuff, else something else\\
	
	- elements, assume worst case that we have a permutation for each leaf in a tree, therefore 5! leaves\\
	- height of any binary tree on 5! leaves will be log(5!) $\approx$ 6.8 $<$ 7\\
	
	
	
	
	
	\section*{Expected Time Analysis}
	
	\section*{Probability Review}
	
	\section*{Intro to Lower Bounds}
	
	
	
\end{document}
