\documentclass[12pt]{article}

\setlength\parindent{0pt}
\newcommand{\myt}[1]{\textbf{\underline{#1}}}

\usepackage{mathtools}
\usepackage{amssymb}
\usepackage{tikz}

\title{\vspace{-15ex}CS240 Tutorial 3\vspace{-1ex}}
\date{May 20th, 2015}
\author{Graham Cooper}

\begin{document}
	\maketitle
	
	\section*{Heaps}
	Consider the following heap (max-heap):\\
	\begin{center}\begin{tikzpicture}[
		level distance=45 pt,
		every node/.style={circle,draw},
		level 1/.style={sibling distance=200 pt},
		level 2/.style={sibling distance=100 pt},
		level 3/.style={sibling distance=60 pt}
		]
		\node {5}
		child {node {5}
			child {node {1}}
			child {node {3}}
		}
		child {node {2}
			child {node {2}}
			child {node {1}}
		};
		\end{tikzpicture}\end{center}
	A: [5,5,2,1,3,2,1]\\
	
	\subsection*{Insert}
	Suppose we insert 17 and 8 into the heap\\
	
	\begin{center}\begin{tikzpicture}[
		level distance=45 pt,
		every node/.style={circle,draw},
		level 1/.style={sibling distance=200 pt},
		level 2/.style={sibling distance=100 pt},
		level 3/.style={sibling distance=60 pt}
		]
		\node {5}
		child {node {5}
			child {node {1}
				child {node {17}}
				child [missing]	
			}
			child {node {3}}
		}
		child {node {2}
			child {node {2}}
			child {node {1}}
		};
		\end{tikzpicture}\end{center}
	\begin{center}\begin{tikzpicture}[
		level distance=45 pt,
		every node/.style={circle,draw},
		level 1/.style={sibling distance=200 pt},
		level 2/.style={sibling distance=100 pt},
		level 3/.style={sibling distance=60 pt}
		]
		\node {5}
		child {node {5}
			child {node {17}
				child {node {1}}
				child [missing]	
			}
			child {node {3}}
		}
		child {node {2}
			child {node {2}}
			child {node {1}}
		};
		\end{tikzpicture}\end{center}
	\begin{center}\begin{tikzpicture}[
		level distance=45 pt,
		every node/.style={circle,draw},
		level 1/.style={sibling distance=200 pt},
		level 2/.style={sibling distance=100 pt},
		level 3/.style={sibling distance=60 pt}
		]
		\node {5}
		child {node {17}
			child {node {5}
				child {node {1}}
				child [missing]	
			}
			child {node {3}}
		}
		child {node {2}
			child {node {2}}
			child {node {1}}
		};
		\end{tikzpicture}\end{center}
	\begin{center}\begin{tikzpicture}[
		level distance=45 pt,
		every node/.style={circle,draw},
		level 1/.style={sibling distance=200 pt},
		level 2/.style={sibling distance=100 pt},
		level 3/.style={sibling distance=60 pt}
		]
		\node {17}
		child {node {5}
			child {node {5}
				child {node {1}}
				child [missing]	
			}
			child {node {3}}
		}
		child {node {2}
			child {node {2}}
			child {node {1}}
		};
		\end{tikzpicture}\end{center}
	
	For 8:
	
	\begin{center}\begin{tikzpicture}[
		level distance=45 pt,
		every node/.style={circle,draw},
		level 1/.style={sibling distance=200 pt},
		level 2/.style={sibling distance=100 pt},
		level 3/.style={sibling distance=60 pt}
		]
		\node {17}
		child {node {5}
			child {node {5}
				child {node {1}}
				child {node {8}}	
			}
			child {node {3}}
		}
		child {node {2}
			child {node {2}}
			child {node {1}}
		};
		\end{tikzpicture}\end{center}
	\begin{center}\begin{tikzpicture}[
		level distance=45 pt,
		every node/.style={circle,draw},
		level 1/.style={sibling distance=200 pt},
		level 2/.style={sibling distance=100 pt},
		level 3/.style={sibling distance=60 pt}
		]
		\node {17}
		child {node {5}
			child {node {8}
				child {node {1}}
				child {node {5}}	
			}
			child {node {3}}
		}
		child {node {2}
			child {node {2}}
			child {node {1}}
		};
		\end{tikzpicture}\end{center}
	\begin{center}\begin{tikzpicture}[
		level distance=45 pt,
		every node/.style={circle,draw},
		level 1/.style={sibling distance=200 pt},
		level 2/.style={sibling distance=100 pt},
		level 3/.style={sibling distance=60 pt}
		]
		\node {17}
		child {node {8}
			child {node {5}
				child {node {1}}
				child {node {5}}	
			}
			child {node {3}}
		}
		child {node {2}
			child {node {2}}
			child {node {1}}
		};
		\end{tikzpicture}\end{center}
	Do not swap with 17.\\
	
	\subsection*{Delete-max}
	Suppose now we want to delete-max:\\
	\begin{center}\begin{tikzpicture}[
		level distance=45 pt,
		every node/.style={circle,draw},
		level 1/.style={sibling distance=200 pt},
		level 2/.style={sibling distance=100 pt},
		level 3/.style={sibling distance=60 pt}
		]
		\node {17}
		child {node {8}
			child {node {5}
				child {node {1}}
				child {node {5}}	
			}
			child {node {3}}
		}
		child {node {2}
			child {node {2}}
			child {node {1}}
		};
		\end{tikzpicture}\end{center}
	1) Swap root with the right most leaf, (on bottom level)\\
	2) remove the largest element\\
	3) Bubble down the new root\\
	
	\begin{center}\begin{tikzpicture}[
		level distance=45 pt,
		every node/.style={circle,draw},
		level 1/.style={sibling distance=200 pt},
		level 2/.style={sibling distance=100 pt},
		level 3/.style={sibling distance=60 pt}
		]
		\node {5}
		child {node {8}
			child {node {5}
				child {node {1}}
				child [missing]	
			}
			child {node {3}}
		}
		child {node {2}
			child {node {2}}
			child {node {1}}
		};
		\end{tikzpicture}\end{center}
	\begin{center}\begin{tikzpicture}[
		level distance=45 pt,
		every node/.style={circle,draw},
		level 1/.style={sibling distance=200 pt},
		level 2/.style={sibling distance=100 pt},
		level 3/.style={sibling distance=60 pt}
		]
		\node {8}
		child {node {5}
			child {node {5}
				child {node {1}}
				child [missing]	
			}
			child {node {3}}
		}
		child {node {2}
			child {node {2}}
			child {node {1}}
		};
		\end{tikzpicture}\end{center}
	
	\subsection{Heapify:}
	
	\begin{enumerate}
		\item n \- size(A) \- 1
		\item for i = floor(n/2) down to 1
		\item bubble-down(A, i)
	\end{enumerate}
	\begin{center}\begin{tikzpicture}[
		level distance=45 pt,
		every node/.style={circle,draw},
		level 1/.style={sibling distance=200 pt},
		level 2/.style={sibling distance=100 pt},
		level 3/.style={sibling distance=60 pt}
		]
		\node {1}
		child {node {6}
			child {node {2}
				child {node {1}}
				child {node {10}}	
			}
			child {node {8}
				child {node {20}}
				child [missing]	
			}
		}
		child {node {7}
			child {node {14}}
			child {node {5}}
		};
		\end{tikzpicture}\end{center}
	
	\begin{center}\begin{tikzpicture}[
		level distance=45 pt,
		every node/.style={circle,draw},
		level 1/.style={sibling distance=200 pt},
		level 2/.style={sibling distance=100 pt},
		level 3/.style={sibling distance=60 pt}
		]
		\node {1}
		child {node {6}
			child {node {10}
				child {node {1}}
				child {node {2}}	
			}
			child {node {8}
				child {node {20}}
				child [missing]	
			}
		}
		child {node {7}
			child {node {14}}
			child {node {5}}
		};
		\end{tikzpicture}\end{center}
	\begin{center}\begin{tikzpicture}[
		level distance=45 pt,
		every node/.style={circle,draw},
		level 1/.style={sibling distance=200 pt},
		level 2/.style={sibling distance=100 pt},
		level 3/.style={sibling distance=60 pt}
		]
		\node {1}
		child {node {20}
			child {node {10}
				child {node {1}}
				child {node {2}}	
			}
			child {node {6}
				child {node {8}}
				child [missing]	
			}
		}
		child {node {7}
			child {node {14}}
			child {node {5}}
		};
		\end{tikzpicture}\end{center}
	\begin{center}\begin{tikzpicture}[
		level distance=45 pt,
		every node/.style={circle,draw},
		level 1/.style={sibling distance=200 pt},
		level 2/.style={sibling distance=100 pt},
		level 3/.style={sibling distance=60 pt}
		]
		\node {1}
		child {node {20}
			child {node {10}
				child {node {1}}
				child {node {2}}	
			}
			child {node {6}
				child {node {8}}
				child [missing]	
			}
		}
		child {node {14}
			child {node {7}}
			child {node {5}}
		};
		\end{tikzpicture}\end{center}
	\begin{center}\begin{tikzpicture}[
		level distance=45 pt,
		every node/.style={circle,draw},
		level 1/.style={sibling distance=200 pt},
		level 2/.style={sibling distance=100 pt},
		level 3/.style={sibling distance=60 pt}
		]
		\node {20}
		child {node {10}
			child {node {1}
				child {node {1}}
				child {node {2}}	
			}
			child {node {6}
				child {node {8}}
				child [missing]	
			}
		}
		child {node {14}
			child {node {7}}
			child {node {5}}
		};
		\end{tikzpicture}\end{center}
	\begin{center}\begin{tikzpicture}[
		level distance=45 pt,
		every node/.style={circle,draw},
		level 1/.style={sibling distance=200 pt},
		level 2/.style={sibling distance=100 pt},
		level 3/.style={sibling distance=60 pt}
		]
		\node {20}
		child {node {10}
			child {node {2}
				child {node {1}}
				child {node {1}}	
			}
			child {node {6}
				child {node {8}}
				child [missing]	
			}
		}
		child {node {14}
			child {node {7}}
			child {node {5}}
		};
		\end{tikzpicture}\end{center}
	
	Overall we bubbled down 2, 6, 7, 6 and then 1\\
	
	\subsection*{Heapsort}
	
	Use heapsort to sort the following array:\\
	A: [2520, 1982, 34]800, 34000, 322,159,2845, 9]\\
	\begin{enumerate}
		\item H $\leftarrow$ Heapify(A)
		\item call delete-max(H) n times and store the result in A.
	\end{enumerate}
	
	\section*{Stack}
	Q: How can we simulate a stack using a priority queue (heap).\\
	
	A stack should support push and pop\\
	
	\begin{verbatim}

	Stack {
	- max-heap H
	- priority P
	}
	
	Push(e) {
	    H.heapinsert(p,e);
	    p = p + 1;
	}
	Pop() {
	    H.deletemax();
	    p = p-1;
	}
	\end{verbatim}
	
	Q: Given K sorted lists, where the combination of all k lists has n elements, combine them into 1 sorted list O(nlogk time. Hint: use priority queues.\\
	
	First idea (use heapsort) \\
	\begin{itemize}
		\item use heapify O(n)
		\item using delete-max n times gives O(nlog(n)) time which is a bit too much
	\end{itemize}
	
	\begin{tabular}{c | c | c | c | c | c}
		1 & 7 & 12 & 19 & & \\ \hline
		2 & 9 & 14 & 18 & 21 & 23 \\ \hline
		3 & 8 & 16 & & & \\ \hline
		5 & 10 & 17 & 70 & 72 & \\ \hline
	\end{tabular}
	
	Look the the first collumn to create a min-heap. Then delete-min and add on the next element onto the min-heap.\\
	
	Note: Always have k items in our heap(height logk) and repeat n times for a total running time of O(nlogk) time\\

	
	
	
	
\end{document}
