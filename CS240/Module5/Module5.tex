\documentclass[12pt]{article}

\setlength\parindent{0pt}
\newcommand{\myt}[1]{\textbf{\underline{#1}}}

\usepackage{mathtools}
\usepackage{amssymb}
\usepackage{tikz,ifthen,amsmath,amssymb,fancyhdr,comment,lastpage}

\title{\vspace{-15ex}Module 5 - Hashing\vspace{-1ex}}
\date{June 16th, 2015}
\author{Graham Cooper}

\begin{document}
	\maketitle
	\section*{Search in a dictionary of n items}
	\begin{itemize}
		\item Unsorted array $\Theta(n)$
		\item sorted array $\Theta(logn)$
		\item AVL Tree $\Theta(logn)$
		\item B-tree $\Theta(logn)$
	\end{itemize}
	
	\subsection*{Can we do better than $\Theta(logn)?$}
	\subsubsection*{Comparison Based Model}
	NO\\
	In a comparison based model\\
	\begin{itemize}
		\item A search algorithm is a binary decision tree
		\item Each internal node is a comparison
		\item each leaf is a set of actionss
		\item each instance (output) requires different actions
		\item n+1 possible outputs.
	\end{itemize}
	
	A binary decision tree with n+1 leaves\\
	- The height of the tree (the number of required comparisons) is $\Omega(logn)$\\
	
	\subsection{Direct accessing}
	- On implementation of Dictionary A.D.T. we have n item with integer keys which are $<$ M\\
	insert(3,John)\\
	insert(5, andrew)\\
	insert(6, shahini)
	
	\subsection*{Hashing}
	A dictionary of n items with numerical keys\\
	-A hash table of size M, a hash function $\rightarrow$ maps each key to an integer in the range [0,M]\\
	- To search, insert, delete look at the mapped index in a hash table\\
	
	Assume M = 7,\\
	--keys are integers and h(k) = j mod 7
	-- insert(26, "john") $\rightarrow$ h(26) = 5\\
	-- insert(39, "hi") $\rightarrow$ h(39) = 4\\
	insert(21, "blah") $\rightarrow$ h(21) = 0\\
	insert(35, "STFF") $\rightarrow$ h(35) = 0\\
	
	Collision $\rightarrow$ when two keys in dictinoary share the same index in the hash table
	Load factor $\alpha = \frac{n}{M}$\\
	It can be $>$ 1 $<=$ 1, $\alpha$ should be constant (independent of n, m)\\
	
	A table of size m, n changes by insertion deletions,\\
	- When load factor becomes too large - too many collisions, slower operation\\
	- Too small - space waste
	In both cases we should do a rehash\\
	
	\subsubsection*{Collision Handling - Chaining}
	h(k) = k mod 7\\
	insert(39) - h(39) =4\\
	insert(3) - h(3) = 4\\
	insert(36) - h(36) = 1\\
	insert(6) - h(6) = 6\\
	insert(20) - h(20) = 6\\
	
	We create a linked list, the start is 6, then 20\\
	Getting a vlue from the linked list is still O(1) since the load balance maeans there are nota lot of collisions\\
	
	\subsubsection*{Average length of lists}
	$\exists$ are n items with m indices\\
	If the hash function distributes items equally between indices, each bucket has $\frac{n}{m} = \alpha$ items\\
	
	Time  complexity of n\\
	search O(1 + $\alpha$)\\
	insert - O(1)\\
	delete = O(1 + $\alpha$)\\
	
	
	\subsection*{Open addressing}
	Each key has a set of candidate indices h(k,0), h(k,1), ... h(k,i)\\
	
	\subsubsection*{Linear Probing}
	Add one to the index until we find an open address.\\
	
	h(k,i) = (h(k) + i) mod M\\
	$\implies$ h(k,0) = k(k)\\
	h(k,1) = h(k) + 1\\
	
	M = 7, h(k) = k mod 7
	insert(6)\\
	insert(4)\\
	insert(0)\\
	insert(2)\\
	insert(17) - h(17) = 3\\
	insert(7) - h(7) = 0\\
	insert(7,b) = h(7) = 0 (is it available, no)\\
	h(7,1) = (h(7) + 1) mod 7 = 1 (good)\\
	insert(16) - h(16,0) = h(16) = 2 (bad)\\
	insert(16,1) - (h(16) + 1) = 3 (bad)\\
	insert(16,2) - h(16) + 2 = 4 - good!\\
	
	M = 7, h(k) = k mod 7\
	insert(2)\\
	insert(3)\\
	insert(9)\\ - h(9) = 2\\
	delete(3)\\
	search(9)\\
	
	We need to mark things as deleted so we do not stop at deleted indexes when we are searching for 9, so that we do not think that they key does not exist as we advance\\
	
	Requires that n $\leq$ m $\implies$ $\alpha \leq 1$\\
	
	\subsubsection*{Double Hashing or Two Hash function}
	
	first candidate h(k,0) = ($h_1$(k)) mod m\\
	Second candidate h(k,1) = ($h_1(k,1)$ + $h_2(k)$) mod m\\
	third candidate $h(k,2)$ = ($h_1(k) + 2h_2(k)$) mod m\\ 
	...\\
	h(k,i) = ($h_1(k) + ih_2(k))$ mod m\\
	
	m = 7, $h_1$(k) = k mod 7, $k_2$(k) = (2k) mod 7\\
	
	insert(4)\\
	- h(k,0) = $h_1(k) = 4 mod 7 = 4$\\
	insert(9)\\
	- h(k,0) = $h_1(9) = 2$\\
	insert(13) \\
	- h(13,0) = $h_1(13) = 6$\\
	insert(24) \\
	- h(k,0) = $h_1(24) = 3$\\
	insert(16)\\
	- h(16,0) = $h_1(16) = 2$ THERE IS A COLLISION\\
	- h(16,1) = $h_1(16) + h_2(16) mod 7 = 2 + 32 mod 7 = 2 + 4 = 6$ Collision again\\
	- h(16,2) = $h_1(16) + 2h_2(16) = 2 + 2 * 4 = 10 mod 7 = 3$ Collision again\\
	- h(16,3) = $h_1(16) + 3h_2(16) = 2 + 3 * 4 = 14 mod 7 = 0$ Good to go\\
	
	Delete(24)\\
	- h(24,0) = $h_1(24) = 24 mod 7 = 3$ Found 24, mark as deleted\\
	Search(16)\\
	- h(16,0) = 2 $-> 9 \neq 16$ not found\\
	- h(16,1) = 6 $-> 13 \neq 16$ not found\\
	- h(16,2) = 3 $-> deleted$ keep searching!\\
	- h(16,3) = 0 $-> 16 = 16$ we found it!!\\
	
	\subsection*{Chaining vs Open Addressing}
	\subsubsection*{Ease of implementation}
	chaining is easier to implement
	\subsubsection*{Space Efficiency, Balance Factor}
	chaining: $\alpha$ can be larger than 1, so space in the hash table is not wasted, but pointers are taking up memory\\
	Open addressing: $\alpha$ must be less than one so there are always going to be empty spaces in the table but there are no pointers taking up memory\\
	
	\subsubsection{Speed}
	Double hashing is generally faster than Linear Probing\\
	
	\begin{tabular}{c | c | c | c | c}
		$\alpha$ & 50\% & 60\% & 75\% & 90\% \\ \hline
		LP - search & 1.5 & 2 & 3 & 5.5 \\
		LP - Insert & 2.5 & 5 & 8.5 & 55.5 \\
		DH - Search & 1.4 & 1.6 & 1.8 & 2.6 \\
		DH - Insert & 1.5 & 2 & 3 & 5.5 \\
	\end{tabular}
	
	\subsection*{Cuckoo Hashing}
	There are two hash functions $h_1, h_2$ then there are two candidate positions for each key, ie, $h_1(k), h_2(k)$\\
	
	To insert a key k\\
	- Place k at index $h_1(k)$\\
	- if required remove key k' from $h_1(k)$\\
	- - place k' in the other candidate index for it\\
	- - this might require removing another k'' and repeating\\
	
	k = 7, $h_1$(k) = k mod 7, $h_2$(k) = 2k mod 7\\
	
	insert(8)\\
	- $h_1$(8) = 1\\
	insert(13)\\
	- $h_1$(13) = 6\\
	insert(9)\\
	- $h_1$(9) = 2\\
	insert(15)\\
	- $h_1$(15) = 1 put 15 at index 1 and kick out 8
	- $h_2(8)$ = 16 mod 7 = 2 place 8 at index 2 and kick out 9\\
	- $h_2(9)$ = 18 mod 7 = 4 place 9 at index 4\\
	
	Good for search/delete since there are only two indices to check\\
	Insertion is a bit messy as there might be a loop after kicking out n keys (rehash)\\
	
	\section*{Choosing a Hash Function}
	Goals:
	\begin{itemize}
		\item scramble the keys
		\item effiently computable
		\item each index is equally likely
	\end{itemize}
	
	If we are hashing our SIN number, the first 3 digits are the same based on certain things such as immigration etc. Last 3 digits would be better because they are seemingly random\\
	
	Date of birth:\\
	Birth year - very bad! very few indexes\\
	Birth day - better, since there are a possible 365 values\\
	
	Phone Numbers\\
	First three digits is bad b/c everyone is from ontario, many 519\\
	Last 3 digits is much better\\
	
	h(k) = floor(M(kA - floor(kA))), where A usually is $\phi$ and M making M to be a power of 2\\
	
	\section*{Hashing vs BSTs}
	(on average)\\
	\begin{tabular}{ c | c | c}
		Function & Hashing & Binary Search Trees \\ \hline
		Search & $\Theta(1)$ & $\Theta(Logn)$\\
		Insert & $\Theta(1)$ & $\Theta(logn)$ \\
		Delete & $\Theta(1)$ & $\Theta(logn)$\\
		Indices are Empty & Memory Wasted & No memory wasted (empty pointer) \\
	\end{tabular}
	
	\section*{External Memory}
	In linear probing, little disk acesses.
	
	In a huge hash table there is (L + 1)/2 Disk Acesses (where L is the lenght of the linked list)
	
	\section*{Terminology}
	\begin{itemize}
		\item L $\rightarrow$ length of hash values in binary
		\item Directory $\rightarrow$ a table of lenght $2^d$ where d $<$ L, di is called the order of extendible hash table
		\item Each index in the directory table has a pointer to a page in external memory (Block of size S)
	\end{itemize}
	
	Each block B has a "local depth $k_B$\\
	- All has values inside the block have the same $k_B$ leading digits $k_B \leq d$\\
	
	See slides. I didn't really feel like taking notes here...\\
	
	\section*{Extendable Hashing}
	\begin{itemize}
		\item S $\rightarrow$ block sizes
		\item d $\rightarrow$ directory order
		\item L $\rightarrow$ hash values length in binary
		\item $K_B \rightarrow$ local depth
	\end{itemize}
	
	insert(15) $\rightarrow$ h(15) = 15 = 01111$_{(2)}$ - the first d = 2 digits are 01\\
	look for item 01 in directory - bring B2 to the main memory\\
	
	insert(16) $\rightarrow$ h(16) = 16 = 10000$_{(2)}$ - 10\\
	- Bring B3 to memory\\
	KB = 1 $<$ 2 = d $\rightarrow$ split B3 into two locks with KB incrememneted\\
	
	
	
	
\end{document}
